\documentclass[11pt,a4paper]{article}
\usepackage[utf8]{inputenc}
\usepackage[T1]{fontenc}
\usepackage{geometry}
\geometry{margin=1in}
\usepackage{parskip}
\usepackage{amsmath,amssymb}
% NOTE: This .tex was converted from PDF; equations and symbols may need manual review.

\title{Structural Reinforcement Learning for Heterogeneous Agent Macroeconomics}
\author{Yucheng Yang, Chiyuan Wang, Andreas Schaab, Benjamin Moll}
\date{Preliminary, December 2025}

\begin{document}
\maketitle

\begin{abstract}
We present a new approach to formulating and solving heterogeneous agent models with aggregate risk. We replace the cross-sectional distribution with low-dimensional 
prices as state variables and let agents learn equilibrium price dynamics directly from simulated paths. To do so, we introduce a structural reinforcement learning (SRL) method which 
treats prices via simulation while exploiting agents’ structural knowledge of their own individual dynamics. Our SRL method yields a general and highly efficient global solution 
method for heterogeneous agent models that sidesteps the Master equation and handles 
models traditional methods struggle with, like those with nontrivial market-clearing conditions. We illustrate the approach in the Krusell-Smith model, the Huggett model with 
aggregate shocks, and a HANK model with a forward-looking Phillips curve, all of which 
we solve globally within minutes. 
\end{abstract}

\section{Introduction}

Many of the most important questions in macroeconomics call for studying models with heterogeneous agents and aggregate risk. A well-known difficulty is that, in standard recursive

rational-expectations formulations of such models, the cross-sectional distribution of agents

becomes a state variable in the Bellman equation characterizing agents’ decision problems –

the “Master equation.”1 This difficulty arises even though agents do not directly “care about”

the distribution, i.e. it does not enter their objective functions; instead, as is standard in competitive equilibrium models, they only care about prices. Intuitively, low-dimensional equilibrium prices do not follow a Markov process but the extremely high-dimensional distribution

does. Therefore, agents with rational expectations forecast prices by forecasting distributions.

While the recent literature has made some impressive advances, the extreme curse of dimensionality inherent in the Master equation remains a central computational bottleneck for

global solutions to heterogeneous agent models with aggregate risk.2 Even seemingly simple

model environments like a Huggett (1993) model with aggregate risk (in which there is a nontrivial market clearing condition) or a one-asset HANK model with a forward-looking Phillips

curve lead to exceedingly difficult computational problems. This lack of a general and efficient global solution method limits the applicability of heterogeneous-agent macroeconomics,

in particular to questions in which aggregate non-linearities play a key role.3

The contribution of this paper is to develop an approach that sidesteps the Master equation

by using ideas from reinforcement learning (RL). RL means learning value or policy functions

of incompletely-known Markov decision processes via some form of Monte Carlo simulation

(Sutton and Barto, 2018). We apply this idea to equilibrium prices and let agents compute price

expectations directly from simulated paths. However, our approach differs from standard

RL in that we assume that agents have structural knowledge about the dynamics of their own

individual states (e.g. their budget constraint and idiosyncratic income process). We term this

hybrid approach structural reinforcement learning (SRL) and our specific algorithm a structural

policy gradient (SPG) algorithm.4 By imposing that policy functions depend only on current

prices (or a short price history) we keep the state space low-dimensional so that we can work

with a grid-based (tabular) approach rather than deep neural networks. Finally, we provide an

efficient implementation in JAX (Bradbury et al., 2018; Sargent and Stachurski, 2025) that can

be run on Google Colab.5

SRL delivers a new and highly efficient global solution method for heterogeneous agent

models with aggregate risk. Importantly, it solves problems traditional methods struggle with.

We demonstrate this with two example applications. First, a model environment with a non1 See, e.g., Den Haan (1996), Krusell and Smith (1998), Schaab (2020), and Bilal (2023). The name “Master

equation” comes from the mathematics literature on Mean Field Games (Cardaliaguet et al., 2019).

2 There is, of course, also the global solution method of Krusell and Smith (1998) and Den Haan (1996) which

assumes that agents forecast prices by forecasting moments of cross-sectional distributions rather than the distributions themselves. We discuss similarities and differences to our approach further below.

3 While we do not consider such models in the present draft, we think that one particularly promising application of our global solution method will be modeling infrequent but large boom-bust cycles like financial crises.

4 The “structural” in SRL is analogous to that in structural vector autoregressions (SVARs).

5 Google Colab is a cloud computing platform that is easily accessible to all researchers.

trivial market-clearing conditions, namely a Huggett (1993) model with aggregate risk. Second, a HANK model with a forward-looking Phillips curve. Both model environments have

proven notoriously difficult for traditional methods.6 We instead solve the Huggett model in

around one minute and the HANK model in around three minutes.7 For completeness, we

also solve the easier Krusell and Smith (1998) model in around 55 seconds.

Almost all existing global solution methods for heterogeneous agent models use dynamic

programming. They either use dynamic programming to directly tackle the high-dimensional

Master equation with the distribution as state variable; or, as in Krusell and Smith (1998), they

use model-generated data to estimate a low-dimensional Markovian “perceived law of motion” (PLM) for moments of the distribution and then apply dynamic programming to this

lower-dimensional approximate problem.8 Our SPG algorithm instead works directly with

the sequential formulation of the problem and never attempts to force it into the standard

Markovian structure required for applying dynamic programming. While we reduce the dimensionality of the state space in a fashion reminiscent of moment-based methods, we never

estimate a PLM. Instead, we use the simple idea at the core of all RL methods that value functions are expected values – here, expected discounted lifetime utilities – and can therefore be

approximated by averaging across simulated trajectories. We then find the low-dimensional

policy function that maximizes expected lifetime utility using stochastic gradient ascent.

The key elements of our SPG method delivering fast computations even in challenging

model environments are as follows. Most importantly, as already mentioned, we replace the

cross-sectional distribution with low-dimensional prices as state variables. We do this in two

steps. The first step is to impose that agents observe the history of equilibrium prices but not

the cross-sectional distribution. By the Wold representation theorem, this assumption does

not, by itself, imply any departure from full information rational expectations. In the second

step, we restrict the agents’ state space: we assume that their policy functions depend only on

current prices or, perhaps, a short price history. This second assumption means that we do

not solve for the model’s rational-expectations equilibrium; instead, we solve for a restricted

perceptions equilibrium (RPE) in the sense of Sargent (1991), Evans and Honkapohja (2001),

or Branch (2006): while agents’ expectations are restricted, they are nevertheless statistically

consistent with actual equilibrium outcomes, thereby fulfilling one of the desiderata of rational

expectations.

The next key element delivering fast computations is the defining assumption of our SPG

approach that agents have structural knowledge about the dynamics of their own individual

6 There is an interesting historical note regarding the computational difficulty of the Huggett model with aggregate risk. According to Maliar and Maliar (2020), in the influential JEDC special issue on numerical solution

methods for HA models with aggregate risk (Den Haan et al., 2010), all participants were initially asked to solve

two benchmark models globally – the Krusell and Smith (1998) model and the Huggett (1993) model with aggregate

shocks (Maliar and Maliar call this “the HANC model with savings through bonds”). However, as they report, no

single team was able to successfully solve the latter model, and it was ultimately dropped from the JEDC project.

7 All experiments are implemented in JAX and are run on a single NVIDIA A100 GPU on Google Colab. We

will discuss the precise convergence criteria in Section 3.7.

8 The PLM is simply an approximate Markov process for the low-dimensional vector of moments. The PLM

may have a simple parametric (e.g. linear) functional form as in Krusell and Smith (1998) or it may be a general,

non-linear function, e.g. parameterized by a neural network as in Fernández-Villaverde et al. (2023).

states. In contrast to standard policy gradient methods which estimate approximate policy gradients, this assumption allows us to compute exact policy gradients by differentiating through

these individual dynamics. The only part of the environment that is treated as unknown is the

process for general-equilibrium prices and aggregate shocks, which is learned from simulated

data. Specifically, our SPG method discretizes the individual state space so that agents’ individual states evolve according to a known transition matrix Aπ where π denotes the vector

of individual policies. When computing policy gradients of lifetime utilities with respect to

π, we exploit this structural knowledge and differentiate through Aπ while using simulation

only for prices and aggregate shocks.

Finally, our low-dimensional grid- and price-based policy functions allow us to (globally)

simulate the economy forward in time in a very efficient manner. For any particular trajectory of aggregate shocks, we update the distribution using the “histogram method” of Young

(2010). The fact that policy functions depend on current prices allows us to efficiently handle

non-trivial market clearing conditions like in Huggett (1993). Intuitively, policy functions double

as individual supply schedules. Integrating across the distribution to obtain the corresponding

aggregate supply schedule, it is then straightforward to solve for equilibrium prices at each

point in time along a simulation path. Our treatment of market clearing mirrors the RL literature’s distinction between agents and environment: agents can interact with their environment

under any given policy including suboptimal ones; finding optimal policies is conceptually

separate. In line with this dichotomy, we treat market clearing as part of the environment and

find equilibrium prices also for suboptimal policies. This approach differs from standard practice in macroeconomics which first finds optimal policies given prices in an inner loop and

then finds equilibrium prices in an outer loop.

In summary, SRL enables efficient global solutions of heterogeneous-agent models with full

cross-sectional distributions. Importantly, while our restricted state space (and use of an RPE)

simplifies agents’ decision problems inside the model, it does not diminish the rich dynamics of

the economy which still evolves stochastically and non-linearly, driven by the policy functions

of forward-looking heterogeneous agents.

We illustrate our method in three benchmark environments: a Huggett (1993) economy,

the classic Krusell and Smith (1998) model, and a one-account heterogeneous agent New Keynesian (HANK) model with sticky prices. In all three cases, our price-based SRL algorithm

converges quickly on modern hardware: solving the Krusell-Smith model takes 55 seconds

while the Huggett and HANK models – typically viewed as more challenging because of

their non-trivial market-clearing conditions and, in the HANK case, a forward-looking Phillips

curve – take only modestly longer (about 75 seconds and roughly 3 minutes, respectively). We

present extensive tests to verify the accuracy of our solutions. For the Krusell-Smith model, our

method produces equilibrium dynamics that closely match the rational expectations obtained

using deep-learning-based methods. As a complementary exercise, we extend the information

set to allow agents to condition on lagged prices and show that this richer price history does

not meaningfully change the solution, suggesting that current prices already contain most of

the information that matters for behavior. Finally, in the HANK application we show how to

use our approach to solve the household and firm problems jointly, using the same SPG algorithm not only for consumption-saving decisions but also for the forward-looking price-setting

problem of firms.

Relation to Economics Literature. A huge theoretical and quantitative literature studies environments in which heterogeneous households are subject to uninsurable idiosyncratic shocks.

See Krusell and Smith (2006), Heathcote et al. (2009), Quadrini and Rı́os-Rull (2015), Krueger

et al. (2016), Sargent (2023), and Auclert et al. (2025a) for surveys.

Within this literature, a sizable subliterature is concerned with the global solution of heterogeneous agent models with aggregate risk. A first strand of the literature tackles the Master

equation directly, typically using deep neural networks – see e.g. Han et al. (2021), Maliar et

al. (2021), Azinovic et al. (2022), Kahou et al. (2021), Duarte et al. (2024), Huang (2023), Kase et

al. (2024), Gu et al. (2024), Payne et al. (2024), and Gopalakrishna et al. (2024). Our approach

differs from all of these papers in that we sidestep the Master equation rather than attempting

to “tame the curse of dimensionality” inherent in it. Among these papers, Han et al. (2021)

is most related to and influential for our work: like them, we learn value and policy functions of heterogeneous agents via simulation; also like them, we take advantage of agents’

structural knowledge of their own individual dynamics – what we have termed a “structural”

RL approach. The key difference is that Han et al. (2021) include the high-dimensional crosssectional distribution in the agents’ state space whereas we replace this distribution with lowdimensional prices as state variables.

A second strand of the literature estimates a low-dimensional Markovian PLM for moments of the distribution and then applies dynamic programming as in Krusell and Smith

(1998) and Den Haan (1996). Similar to us, this approach sidesteps the Master equation by

working with a low-dimensional state space that does not include the distribution. One variant of this approach formulates PLMs directly in terms of equilibrium prices so that – like

in our approach – this low-dimensional state space includes prices.9 However, this approach

introduces an additional fixed-point loop in which the perceived law of motion is repeatedly

updated and the associated dynamic programming problem re-solved, a procedure that is

computationally slow in many economically interesting environments, particularly those with

non-trivial market-clearing conditions. By contrast, we work directly with the problem’s sequential formulation and never estimate a PLM.

Our approach is also related to the adaptive learning literature (e.g. Bray, 1982; Marcet and

Sargent, 1989; Evans and Honkapohja, 2001). Jacobson (2025) and Giusto (2014) apply adaptive learning in heterogeneous agent models with aggregate risk. RL and adaptive learning are

linked because both are special cases of a more general set of stochastic approximation methods (Robbins and Monro, 1951). Similar to the learning literature, our approach computes an

equilibrium that is “self-confirming” (Sargent, 1999; Cho and Sargent, 2016).10 But because

9 See for example Lee and Wolpin (2006), Storesletten et al. (2007), Gomes and Michaelides (2008), Favilukis et

al. (2017), Llull (2018), Kaplan et al. (2020), and Fernández-Villaverde et al. (2024b).

10 Agents form price expectations from data generated by the economy in which they live. Their expectations

are therefore statistically consistent with actual equilibrium outcomes but may be incorrect for events that are

we restrict the individual state space to not include the distribution, our self-confirming equilibrium is an RPE (Sargent, 1991; Branch, 2006). In the language of Guarda (2025), our RPE

features expectations that are both “narrow” and “short”: narrow because they do not condition on the distribution and short because they do not condition on a long price history.11

Our focus on a low-dimensional vector of equilibrium variables links SRL to the “sequence

space” approach of Auclert et al. (2021) and Auclert et al. (2025b). The difference is that SRL

handles stochastic price sequences outside steady-state neighborhoods. By using Monte Carlo

simulation, it yields a global rather than local solution method in sequence space.12

To be clear, we do not view SRL as an empirically realistic model of human learning.13 Instead, it is an efficient computational method for finding RPEs in heterogeneous agent models

with aggregate risk. Nevertheless, as we note in the conclusion, the core idea of our approach –

agents forming expectations about equilibrium prices by sampling – could, in principle, serve

as a building block for an empirical theory of expectations formation in macroeconomics.

Relation to RL Literature.

Our approach connects to several central ideas in the RL literature.

As already mentioned, in RL, agents learn optimal policies in Markov decision processes using

sampled trajectories (Sutton and Barto, 2018). RL is at the core of some impressive advances

in artificial intelligence, e.g. RL agents learning to play Go and Atari games better than humans (Mnih et al., 2015; Silver et al., 2016, 2017) or post-training of “reasoning” large language

models (LLMs OpenAI, 2024; DeepSeek-AI, 2025).14

Conceptually, our SRL method adopts the RL principle of optimizing policies using Monte

Carlo estimates of expected returns. But it exploits full structural knowledge of each agent’s

reward function and individual transition dynamics to compute exact end-to-end policy gradients. Only equilibrium objects are treated as unknown parts of the environment. In this

sense, our SRL approach sits between model-based and model-free RL: we differentiate exactly through the known micro-level transition probabilities while learning the induced macro

environment from simulation. Also see Han et al. (2021).

While classical RL demonstrates considerable power in solving complex environments, its

general-purpose nature implies suboptimal performance in specialized domains. For instance,

when applied to the computationally intensive post-training phase of LLMs, classical RL algorithms introduce substantial overhead. To address this, Rafailov et al. (2024) reformulated RL

infrequently observed (e.g. events off the equilibrium path).

11 Our notion of equilibrium is also closely related to Guarda’s nonparametric restricted perceptions equilibrium

(NRPE). The difference is that Guarda defines his NRPE recursively and uses dynamic programming whereas our

formulation is sequential.

12 Recent work has explored global solution methods in sequence space using high-dimensional approximation

techniques, e.g., Azinovic-Yang and Žemlička (2025).

13 Moll (2025) proposed three criteria for alternatives to rational expectation in heterogeneous agent models: (1)

computational tractability, (2) consistency with empirical evidence, and (3) (some) immunity to the Lucas critique.

Our SRL approach delivers on (1) and (3) but not (2).

14 RL ideas have also been applied in economics. Besides Han et al. (2021), see for example Barberis and Jin

(2023), Chen et al. (2023), de la Barrera and de Silva (2024), Calvano et al. (2020), Dou et al. (2025), and the references in Fernández-Villaverde et al. (2024a) for applications in finance and macroeconomics. RL ideas have also

been applied in game theory (e.g. Erev and Roth, 1995, 1998; Fudenberg and Levine, 2016) but using a different

formulation that does not work with value functions and instead directly reinforces the “propensities” of choosing

strategies.

from human feedback (RLHF) as a deep learning problem and proposed the Direct Preference

Optimization (DPO) algorithm, which significantly reduces computational cost and enhances

training stability. Similarly, Hu et al. (2020) and Freeman et al. (2021) introduced differentiable physics engines – DiffTaichi and Brax (the latter also implemented using JAX) – that

enable gradient computation directly from built-in physical equations.15 This allows policy

updates via exact gradient methods, eliminating the need for sampling-based approximations

and thereby improving both accuracy and efficiency. In Brax, this method is termed “analytical

policy gradient,” which shares conceptual foundations with our SRL framework.

Heterogeneous-agent models in macroeconomics can be viewed as a special case of the

more general Mean Field Games (MFGs) studied in the mathematics literature (e.g. Lasry and

Lions, 2007; Cardaliaguet et al., 2019). In the RL literature, the work closest to ours is therefore

the line of research that applies RL methods to solve MFGs (e.g. Yang et al., 2018; Laurière et

al., 2022, 2024; Xu et al., 2023; Wu et al., 2025). With the exception of Wu et al. (2025), none

of these papers study the case with aggregate risk (or “common noise” in MFG terminology)

considered here. More importantly, none of the papers applying RL to MFGs exploit the structural knowledge of agents’ individual dynamics. Specifically, they do not exploit the transition

matrix (or infinitesimal generator) for individual states Aπ and its known dependence on the

policy π, even though this matrix (operator) is typically available and even used for updating

the distribution.16

In order to sidestep the Master equation and keep the state space manageable, we assume

a form of partial observability: agents do not observe the high-dimensional distribution which

is the system’s underlying Markov state and instead observe low-dimensional prices. That

being said, a similar SRL approach can also be applied to fully observable MFGs as in Han et

al. (2021).

Roadmap. Section 2 describes the SRL method and Section 3 reports our computational

experiments. Section 4 concludes.

\section{Sidestepping the Master Equation via Structural Reinforcement Learning}

This section describes our SRL method. Rather than solving the Master equation (11) with

the full cross-sectional distribution as a state variable, we work with a low-dimensional state

consisting of prices pt (and the aggregate shock zt ). We adapt RL ideas to let agents learn

optimal behavior from simulated equilibrium data, taking (st , zt , pt ) as their state. In contrast

to standard RL, our SRL method exploits agents' structural knowledge of their own individual

dynamics.

\subsection{The Key Idea of Reinforcement Learning: Monte Carlo instead of Bellman}

Before proceeding, we briefly summarize the basic ideas of RL.17 RL means learning value or

policy functions of incompletely-known Markov decision processes via some form of Monte

Carlo simulation. The key problem addressed by RL is: what to do in dynamic optimization

problems in which the agent does not know the exact environment she is operating in, specifically the stochastic process for the underlying state variables? The key insight of RL is that, in

such environments, one can still approximate optimal value and policy functions as long as one

can simulate.

A simple analogy is how to compute the expected value E[ x ] of a random variable x. The

R

standard way is to compute E[ x ] = x f ( x )dx for a known probability distribution f ( x ). But

what if f is unknown? In such cases, one can often still sample from f and approximate the

expected value E[ x ] with the sample mean x̄ = N1 ∑nN=1 xn .

Building on this intuition, consider the question of how to calculate the following value

function:

"

v0 = E

∞

\#

∑ β u( pt ) ,

t

t =0

where u is a utility function and pt is some exogenous stochastic process. The standard approach is to use dynamic programming: assume that pt is Markov with known transition

probabilities f ( p0 | p); then write and solve the Bellman equation

v( p) = u( p) + β

Z

v( p0 ) f ( p0 | p)dp0 .

An alternative approach is to use Monte Carlo simulation: simply sample N trajectories \{ pnt \}Tt=0

for n = 1, ..., N and approximate the expected value v0 as

v0 ≈ vb0 =

1 N T t

∑ β u( pnt ).

N n∑

=1 t =0

This basic idea – to compute expected values via simulation – lies at the heart of all RL algorithms. Crucially this simulation-based approach does not require knowledge of the transition

probabilities f . It also works directly with the sequential formulation of the problem. In particular, it is unnecessary to force the problem into the standard Markovian structure required

for applying dynamic programming, e.g. by estimating a perceived law of motion for prices

pt . As we explain next, our SRL approach to heterogeneous agent macroeconomics uses this

same approach to compute expectations about equilibrium prices.

17 See Sutton and Barto (2018) and Zhao (2025) for brilliant textbook treatments. Also see Murphy (2025) and

Silver (2015).

\subsection{Revisiting the Agents’ Decision Problem}

Recall that agents choose \{ci,t \} to solve

vi,0 = max E

\{ci,t \}

 ∞

∑ β u(ci,t )

t



s.t.

si,t+1 ∼ Ts (· | si,t , ci,t , zt , pt ),

pt = P∗ ( Gt , zt ).

(12)

t =0

We begin by specifying what individuals observe and on what they condition their decisions.

Assumption 1 (Information). At date t, an individual observes the entire history of aggregate prices

\{ pt , pt−1 , . . .\}, but not the cross-sectional distribution gt .

Assumption 1 rules out direct conditioning on the distribution but does not, by itself, imply

any departure from full information rational expectations. Under standard regularity conditions, a stationary price process \{ pt \} admits a Wold representation and can be written as an

infinite-order vector moving average process

∞

pt = ∑ κ j ε t− j ,

j =0

for some coefficients \{κ j \} and white-noise innovations ε t . Under additional restrictions (invertibility), the stationary price process also has an equivalent VAR(∞) representation,



p t +1 ∼ T p · p t , p t −1 , p t −2 , . . . ,

so that the infinite history of prices is sufficient for forecasting under rational expectations. In

this sense, the price history is rich enough to recover all information that is relevant to the

agents, even though they never observe the cross-sectional distribution directly. Note that the

Wold representation theorem effectively converts the recursive formulation for the price process (10) into a sequential stochastic process. It is also worth pointing out that, in the language

of the RL literature, Assumption 1 is a “partial observability” assumption.

Of course, working with the infinite price history is infeasible in practice. Instead, starting from the rational expectations benchmark in Assumption 1, we restrict attention to lowdimensional summaries of this history. We begin with the most restrictive case, in which agents

condition only on current prices.

Assumption 2 (Restricted state space). Agents’ decision rules (policy functions) take the form

π ( s t , z t , p t ),

so that policies do not depend on lagged prices.18

18 Whether z

t is payoff-relevant depends on the particular application. If, conditional on ( st , pt ), zt affects neither

current payoffs nor individual transitions, then it can be dropped from the state vector for the individual’s decision

problem. For example, zt is directly payoff-relevant in the Huggett model because it scales current

income yi,t zt . By contrast, in our Krusell-Smith and HANK applications in Sections 4.2 and 4.3, zt matters for

individuals only through equilibrium prices, so policies could equivalently be written as π (st , pt ). For notational

uniformity we keep zt in π (st , zt , pt ), but in these cases it is redundant from the household’s point of view.

Assumption 2 imposes a particular restriction on perceptions: agents treat the current price

vector as a sufficient statistic for decision making, even though in the true equilibrium the

price process is not Markov in pt alone. Within this class of policies we then solve problem (12)

by optimizing over many simulated equilibrium paths.

In Section 4.1 we relax Assumption 2 and allow policies to depend on a short history of

past prices. More generally, one could let the policy depend on the hidden state of a recurrent

neural network that summarizes the price history; this fits directly into the same RL-based

framework (Hausknecht and Stone, 2017).

Given any (possibly suboptimal) policy π (s, z, p), the discretization of individual states on

a finite grid $s \in \{s_1, \ldots, s_J\}$ allows us to write the policy in vector form
$\boldsymbol{\pi}(z, p) = (\pi(s_1,z,p), \ldots, \pi(s_J,z,p))^T$.

We represent the cross-sectional distribution as a vector gt , and the individual

transitions induced by the policy are encoded in a sparse transition matrix Aπ (z,p) .

\subsection{Sequential Restricted Perceptions Equilibrium}

Under Assumptions 1 and 2, an individual’s relevant state is simply (st , zt , pt ). Given this

restricted state space, we define equilibrium as a variant of a restricted perceptions equilibrium

(RPE) in the sense of Sargent (1991) and Branch (2006). Because we work directly with the

sequential formulation of the agents’ problem, we term it a “sequential RPE”.

Definition 1 (Sequential restricted perceptions equilibrium). A sequential restricted perceptions

equilibrium consists of a pair of mappings (π ∗ (s, z, p), P∗ (g, z)) such that:

1. Optimality. Given a stochastic process \{zt \}t≥0 as well as a price process \{ pt \}t≥0 generated by

pt = P∗ (gt , zt ), the policy π ∗ = \{c∗ , b0∗ \} ∈ Π solves

max E

π ∈Π

 ∞

∑ β u(c(st , zt , pt ))

t



s.t.

st+1 ∼ Ts (· | st , π (st , zt , pt ), zt , pt ),

t =0

where c(s, z, p) is the consumption component of π (s, z, p). Here Π denotes the set of measurable

policies π : S × Z × P → C × B that satisfy the budget and borrowing constraints.

2. Market clearing. For every t, the market clearing conditions hold: in the Huggett model

Z

b0 (s, zt , pt ) dGt (s) = 0,

(13)

where b0 (s, z, p) is the saving component of π (s, z, p). The solution is a mapping pt = P∗ (gt , zt ).

3. Consistency. When all agents follow π ∗ , the cross-sectional distribution evolves according to

gt+1 = ATπ ∗ (zt ,pt ) gt ,

where Aπ (z,p) is the transition matrix induced by Ts and π and prices are given by pt =

P ∗ (gt , z t ).

This equilibrium notion has three features that are important for our purposes. First, it is a

sequential equilibrium: We work directly with time paths of states and prices rather than with

a recursive formulation in terms of a Markov state. Second, it is self-confirming in the sense

that, given their information and policy, agents’ beliefs about price dynamics are consistent

with the equilibrium price process along the realized paths. Third, it features restricted perceptions because agents condition only on prices, not on the full distribution. This both reflects a

realistic informational environment and greatly reduces the dimensionality of the state space.

As mentioned in the introduction, in the language of Guarda (2025), agents’ beliefs are both

“narrow” and “short”.

These features make it natural to use RL. In practice, we parameterize the low-dimensional

policy function π (s, z, p) on the discretized state space and approximate the maximization

problem (12) by evaluating it along many simulated equilibrium paths. In the next subsection,

we describe how to simulate these trajectories efficiently under a given candidate policy.

\subsection{Simulating the Economy for Given Policy Functions}

Starting from an initial pair (g0 , z0 ), the simulated economy evolves according to

zt+1 ∼ Tz (· | zt )

gt+1 = ATπ (zt ,pt ) gt

p t = P ∗ (gt , z t ).

Updating the aggregate shock zt+1 is straightforward. For the distribution gt+1 , we adopt

Young (2010)’s non-stochastic simulation method and extend it to a full matrix formulation: the

policy function induces a sparse transition matrix over the grid, and the distribution evolves

deterministically via matrix-vector multiplication. The main computational difficulty lies in

the last step. In some models, such as Krusell and Smith (1998), the price functional P∗ (gt , zt )

is available in closed form. In others, such as the Huggett model, prices are

defined only implicitly by market-clearing conditions. Computing such implicit prices represents a significant challenge. Most existing numerical methods solve a non-linear root-finding

problem in every period of the simulation, typically making this step the slowest part of the

algorithm; see for example Krusell and Smith (1997), Schaab (2020), as well as the historical

note in footnote 6 in the introduction. Our method delivers an efficient way to compute these

implicit prices along simulated paths. We show next how this works in the Huggett economy

there.

Efficient handling of non-trivial market clearing conditions. In the Huggett model, the

equilibrium interest rate pt = rt is pinned down by the requirement that bonds must be in

zero net supply, see (6). The key insight that allows us to find equilibrium prices efficiently

is that, due to Assumption 1, individual policy functions π (s, z, p) depend on current prices p

rather than the cross-sectional distribution Gt (s). In addition to the much lower dimensionality

of p as compared to Gt (s), this has another key payoff: policy functions double as individual supply schedules which can easily be aggregated to obtain aggregate supply curves at each point in

time which also depend on the current price p. Given an aggregate supply curve as a function

of p, it is then straightforward to solve for the equilibrium price pt = P∗ ( Gt , zt ).

To see this in more detail, consider the market clearing condition in our restricted perceptions equilibrium (13). Equivalently,

St ( p t , z t ) = 0

where

St ( p, z) =

Z

b0 (s, z, p)dGt (s)

is aggregate savings implied by the individual saving policy function b0 (s, z, p). The key observation is that the individual saving policy function depends on the interest rate p. This

means that varying p traces out an entire individual supply schedule p 7→ b0 (s, z, p). Aggregating yields the analogous aggregate supply schedule p 7→ St ( p, z). The equilibrium price can

then be computed time-period by time-period along each simulated path. Our approach of

including current prices in the state space to clear markets at each point along a simulation is

similar to Krusell and Smith (2006, Section 3.5).

0.06

z = 0.97, r = 0.031

z = 0.99, r = 0.027

z = 1.00, r = 0.024

z = 1.01, r = 0.021

z = 1.03, r = 0.017

z = 1.05, r = 0.017

Total assets B

Interest rate r

0.05

0.04

0.03

0.02

0.01

0.6

0.4

0.2

0.0

0.2

0.4

Aggregate saving S(r, z)

0.6

0.8

Figure 1: Aggregate saving function S(r, z) in the Huggett economy

Finding the equilibrium interest rate can be numerically implemented in a variety of ways.

In our experiments, we found that the simplest way of doing this is to use the discretized

representation. Once we know the vectors gt and b0 (z, p) for the discretized cross-sectional

distribution and saving policy function, we can compute the entire saving schedule at time t

for all grid values of (z, p) as

St ( p, z) = b0 (z, p)T gt .

(14)

Figure 1 plots the aggregate saving function p 7→ S( p, z) in the Huggett economy at some

date t, with each line corresponding to a grid value of z. Given the realized aggregate state

zt , we select the corresponding supply curve St (·, zt ) and determine the interest rate pt so that

St ( pt , zt ) = 0. In our numerical experiments the function St ( p, z) is weakly increasing in p

under the optimal policy b0 (z, p) and in a neighborhood of the market-clearing price, so the

solution is unique.19 However, market clearing is part of the agents’ environment and needs

to hold for all candidate policy functions, not just at the optimal policy. As we explain in

Section 3.7 below, we solve for optimal policies π (s, z, p) = \{c(s, z, p), b0 (s, p, z)\} iteratively

starting from some initial guess. This means that it is important to have a reasonable initial

guess for b0 (s, z, p) that delivers an upward-sloping aggregate saving supply under this initial

guess.20

Both the dot product in (14) and the interpolation step are simple vector operations and

are very fast when implemented with JAX on GPUs. It is therefore computationally cheap to

compute the entire vector of aggregate savings S( p, z) along each simulated trajectory. Given

this precomputed saving schedule, we use linear interpolation to find the market clearing price

pt in each period and trajectory instead of calling a separate non-linear solver inside an outer

fixed-point iteration as is common in existing methods (Krusell and Smith, 1997; Schaab, 2020).

\subsection{Exploiting Agents’ Structural Knowledge of their Individual Dynamics}

Standard RL applications, such as board games or Atari environments, typically treat both

rewards and state transitions as unknown and learn them entirely from simulated data. Economic models are different. Agents know their preferences and they know how their choices

affect their own individual state next period (Han et al., 2021). We make explicit use of this

structure and refer to our approach as structural reinforcement learning (SRL).

The value for an individual i starting from state (s, z, p) and following a given policy

π (s, z, p) = \{c(s, z, p), b0 (s, z, p)\} is

vπ (s, z, p) = E

 ∞



∑ β u(c(si,t , zt , pt )) si,0 = s, z0 = z, p0 = p .

t

(15)

t =0

Here vπ (s, z, p) is the value associated with a fixed policy π and a given stochastic process for

the aggregate variables ( pt , zt ). The evolution of the individual state si,t is governed by the

known transition kernel Ts or, when individual states are discretized, by the known transition

matrices Aπ (zt ,pt ) . Agents know this mapping from current states and actions into next-period

states because it is implied by their budget constraint and the exogenous process for idiosyn19 On the bounded price grid \{ p \}K

K

k k=1 we compute \{ St ( pk , zt )\}k=1 and find the two adjacent points that bracket

the root. If

S t ( p k , z t ) ≤ 0 ≤ S t ( p k +1 , z t ),

we set the market-clearing price to the linear interpolant

pt = pk −



St ( p k , z t )

p

− pk .

S t ( p k +1 , z t ) − S t ( p k , z t ) k +1

20 For example, in our numerical experiments with power utility u0 ( c ) = c−σ , we have found that the initial







yz

yz 

yz

guess c0 (b, y, z, r ) = 1 + r − ( β(1 + r ))1/σ b + r and b00 (b, y, z, r ) = ( β(1 + r ))1/σ b + r − r works well.

This would be the optimal policy function if all of (y, z, r ) were fixed over time rather than varying stochastically.

One can also use this same initial guess but with a parameter σ0 < σ in place of σ which makes saving more

responsive to the interest rate r and thus the aggregate saving supply better behaved.

cratic income, which we assume agents know, i.e. agents have rational expectations about

their idiosyncratic income process. By contrast, agents do not know the law of motion for the

aggregate variables ( pt , zt ) which are determined in general equilibrium from everyone’s decisions and market clearing. Under rational expectations, one would require agents to know

the stochastic process for ( pt , zt ) exactly and to treat (gt , zt ) as a Markov state so that prices

become a function pt = P∗ (gt , zt ). In our setup, agents instead take the process for ( pt , zt ) as

an object to be learned from simulated data.

SRL takes advantage of this separation between individual and aggregate dynamics by

partitioning the state space (s, z, p) into individual states s and aggregate states (z, p) and treating

these differently. We use the structural model to simulate individual transitions and payoffs

exactly, i.e., conditional on a policy π (z, p), we treat the transition matrix Aπ (z,p) and hence

the mapping (st , zt , pt ) 7→ st+1 as known by the agent and use this matrix to compute the

value function the agent maximizes – see (16) below. We then use RL only to deal with the

low-dimensional but non-Markov aggregate state (zt , pt ).

In particular, our algorithm updates the policy π so as to increase vπ (s, z, p) based on simulated histories of (zt , pt ) rather than using a Bellman equation on the high-dimensional state

space (gt , zt ). This keeps the learning problem low-dimensional while preserving the full

heterogeneous-agent structure of the economy. While the simulations feature the full crosssectional distribution which evolves stochastically and non-linearly over time, we never approximate that distribution nor any mapping from it. Instead only low-dimensional prices

enters the agent’s state.

\subsection{Problem To Be Solved}

The value vπ (s, z, p) associated with a given policy π (s, z, p) was defined in (15). The agent’s

problem is to choose a policy π = (c, b0 ) to maximize her value vπ given an initial state (s, z, p).

It is convenient to use discretized notation and rewrite the value function vπ (s, z, p) in vector

form as

vπ (z, p) = E

 ∞



∑ β Aπ,0→t u(c(zt , pt )) z0 = z, p0 = p ,

t

(16)

t =0

where π (z, p) = \{c(z, p), b0 (z, p)\} is the discretized policy vector and

Aπ,0→t = Aπ (z0 ,p0 ) × · · · × Aπ (zt−1 ,pt−1 )

(17)

denotes the transition matrix of individual states between time 0 and time t under a particular

trajectory \{zτ , pτ \}tτ−=10 . Note again our partitioning of the state space into s and (z, p): the

transition matrices Aπ (zt ,pt ) keep track of all s-transitions while the expectation in (16) is taken

only over (z, p)-trajectories.21 Importantly, the transition matrix Aπ (zt ,pt ) encodes all relevant

structural knowledge about the dynamics of agents’ own individual states s. The presence

of this transition matrix in the optimization objective (16) is why we refer to our approach as

21 Tracking value vectors using the transition matrix A

π in this way is analogous to the use of such matrices in

finite-difference methods for continuous-time HJB equations and HA models (Achdou et al., 2021).

structural RL.

When agents choose policies π they take as given the evolution of equilibrium prices which

evolve according to the true general-equilibrium dynamics,

p t = P ∗ (gt , z t ),

gt+1 = ATπ (zt ,pt ) gt ,

zt+1 ∼ Tz (· | zt ),

(18)

with (g0 , z0 ) given.

The agents’ objective is to find a policy vector π (z, p) that maximizes vπ (z, p) for all initial

values (z, p) taking as given the evolution of equilibrium prices in (18). Note that maximizing

agents take into account the dependence of the transition matrix Aπ,0→t in (16) and (17) on the

policy π; in contrast, because they take prices as given, they do not take into account how price

dynamics depend on the policy π via the term ATπ (zt ,pt ) in the Chapman-Kolmogorov equation

for gt in (18).22

An important feature of this problem is that, while the true state of the economy (gt , zt )

is extremely high-dimensional (because gt is a full cross-sectional distribution), the state that

enters agents’ policy and value functions (st , zt , pt ) is low-dimensional. The agent does not

work with a perceived law of motion for prices. Instead, she treats the process for ( pt , zt )

as given, observes realized sequences along simulated paths, and bases decisions on these

realizations. As a result, there is no inner-outer fixed-point loop over perceived price laws of

motion as in Krusell and Smith (1998). Our algorithm therefore operates directly on the lowdimensional state (st , zt , pt ) from the agent’s perspective, while the high-dimensional state

(gt , zt ) only appears in the background through the law of motion for prices.

\subsection{Implementation: Structural Policy Gradient Algorithm}

To evaluate a candidate policy π, we approximate the value vector using Monte Carlo simulation as explained in Section 3.1. We simulate N trajectories of the economy under this policy

and form the sample analog of the value vector

1 N

v

bπ =

N n∑

=1

 T

∑

t

n

n n

β Aπ,0

→t u (c( zt , pt ))



,

(19)

t =0

where T is a large truncation horizon and the simulated paths are generated from

pnt = P∗ (gtn , znt ),

gtn+1 = ATπ (zn ,pn ) gtn ,

t

t

znt+1 ∼ Tz (· | znt ),

(20)

starting from initial conditions g0n ∼ ψg and z0n ∼ ψz . Thus, v

bπ in (19) is the sample analog of

vπ (z, p) in (16) averaged over the initial distribution of (z, p) induced by the initial distribution

of (g, z), i.e. v

bπ ≈ Ez0 ∼ψz ,g0 ∼ψg [vπ (z0 , p0 )] with p0 = P∗ (g0 , z0 ).

In practice, we work with a scalar objective that averages over initial individual states. Let

d0 denote the uniform distribution on the individual-state grid, so that d0 (s j ) = 1/J. We then

22 As we explain below, in our SPG algorithm which maximizes the Monte Carlo counterpart to (16) with respect

to π, we apply a stop-gradient to simulated prices to ensure that agents take prices as given.

maximize

L(θ ) = dT0 vbπ .

(21)

Because our state space is low-dimensional, we can work with a grid-based (tabular) approach

and we parameterize the policy as



 



π ( z1 , p1 )

π ( s1 , z1 , p1 )



 



..

..

=

,

θ=

.

.



 



π (zK , p L )

π (s J , zK , p L )

where J, K, and L are the numbers of grid points on the s, z, and p grids. That is, the parameter

vector θ is simply the J × K × L-dimensional vector of values of the policy on the (s, z, p) grid.23

Furthermore, in practice it is costly to compute the high-dimensional matrix multiplications

n

Aπ,0

→t = Aπ (z0n ,p0n ) × · · · × Aπ (znt−1 ,pnt−1 ) in (19). After substituting in (21), we therefore rewrite

the optimization objective as

L(θ ) =

1 N T t n T

∑ β (dπ,t ) u(c(znt , pnt )),

N n∑

=1 t =0

n = (An

T

where dπ,t

π,0→t ) d0 is the cross-sectional distribution of s at time t under policy π start-

ing from the initial uniform distribution d0 . We compute this distribution iteratively by solving

n

n

T n

the Chapman-Kolmogorov equation dπ,t

+1 = ( Aπ (zt ,pt ) ) dπ,t forward in time using the Young

(2010) method. 24

Figure 2: Computational graph

Figure 2 presents a computational graph that illustrates the algorithm. Since the mapping

23 One can instead parameterize the policy as a neural network π ( s, z, p; θ ).

For the low-dimensional policy

functions in this paper, a grid-based parameterization is sufficient.

24 Azinovic et al. (2022) use the Young (2010) method in a related fashion to construct an optimization objective

involving a discretized cross-sectional distribution. However, they use neural networks to minimize Euler equation

errors whereas we use a grid-based approach to maximize a value vector.

from parameters θ to the objective L(θ ) is differentiable, we use stochastic gradient ascent (or

variants) to update

θ k+1 = θ k + ηk ∇θ L(θ k ),

where ηk is the learning rate at iteration k. A stop-gradient is applied to the price update

pnt = P∗ (gtn , znt ) in (20) so that, when computing gradients, prices are treated as exogenous.

This is consistent with the standard notion of competitive equilibrium, in which agents take the

process for ( pt , zt ) as given and do not internalize how their behavior affects price formation.

Finally, we assess convergence of the algorithm by tracking the change in the policy vector θ

over the (s, z, p) grid. Convergence is achieved when the L∞ (sup) norm of the policy update

falls below a tolerance,

kθ k+1 − θ k k∞ < econverge

where

kxk∞ ≡ max | x jkl |.

j,k,l

We report our choice of convergence tolerance econverge for each application in Appendix A.

Algorithm 1 summarizes the implementation.

Algorithm 1: Structural Policy Gradient Algorithm

Input : Initial policy parameters θ 0 ; step size sequence \{ηk \}; number of simulated

trajectories N; horizon T.

Output: Approximate optimal policy parameters θ ∗ .

1. Initialize parameters θ 0 .

2. For each iteration k = 0, 1, 2, . . . :

(a) Simulate N trajectories

\{(znt , pnt , gnt )\}tT=0 ,

n = 1, . . . , N,

using policy π (·; θ k ) and market clearing conditions.

(b) Compute the sample objective

L(θ k ) = dT0 vbπ

1 N

where v

bπ =

N n∑

=1

"

T

\#

∑ βt Aπ,0→t u(c(znt , pnt )) .

t =0

(c) Update parameters by stochastic gradient ascent (or variants):

θ k+1 = θ k + ηk ∇θ L(θ k ).

(d) Stop when convergence criteria are met.

\section{Computational Experiments}

In this section we report computational experiments for three benchmark economies: the

Huggett model, the classic Krusell and Smith (1998) model, and a one-account

heterogeneous agent New Keynesian (HANK) model with nominal rigidities.

We implement all three models in JAX and run them on a single NVIDIA A100 GPU on

Google Colab. Table 1 summarizes performance. Since our algorithm is stochastic and uses

Monte Carlo simulations, we run our algorithm 10 times for each specification and report

averages across runs. The first column shows the average number of epochs until convergence,

and the last column the corresponding average time for a single run.

Model

Average converge epoch

\# Runs

Average Runtime (sec)

Krusell-Smith

438.4

56.55

Huggett with agg. shocks

480.6

75.29

HANK with agg. shocks

496.5

199.53

Partial equilibrium (Huggett)

289.3

39.49

Table 1: Runtimes

Solving the Krusell-Smith model takes about 55 seconds, in line with other global solution

methods in the literature. By contrast, the Huggett and HANK models are typically viewed

as more challenging because they feature non-trivial market clearing conditions: standard approaches nest an inner loop that repeatedly solves for prices until markets clear. Nevertheless,

our method solves the Huggett model in 75 seconds and the HANK model in about 3 minutes.

Finally, we also compare the cost of computing the model’s general equilibrium (GE) to

that of computing the corresponding partial equilibrium (PE) problem (see below). We find

that, while computing the GE problem takes longer as expected, the difference in runtime

is modest. In the Huggett model, for example, moving from partial to general equilibrium

increases runtime from 39 seconds to 75 seconds. This is because we do not solve general

equilibrium with a nested inner-outer loop that alternates between solving optimal policies

and updating price functions or perceived laws of motion. Instead, prices are learned in an

online fashion: along each simulated path we compute the market-clearing price implied by

current policies, and the policy update uses these realized prices directly.

\subsection{Huggett Model with Aggregate Risk}

We start our computational experiments with the Huggett economy described below.25

Calibration. We interpret one period as a year and set the household discount factor to β =

1− σ

0.96. Preferences are isoelastic u(c) = 1c −σ and we set σ = 2. In the Huggett model, both the

idiosyncratic and the aggregate income components follow log AR(1) processes. We set the

persistence parameters to ρy = 0.6 and ρz = 0.9 and the standard deviations of the innovations

to νy = 0.2 and νz = 0.02. We discretize these processes on finite grids using a standard

Tauchen procedure (details in the Appendix A). Finally, we impose a borrowing limit b = −1

25 Our comparison of runtimes in Table 1 and Figure 4 below make reference to the partial equilibrium problem

of the Huggett economy. We present the details in Appendix A.1.

and fix aggregate bond supply at B = 0, so bonds are in zero net supply. The full calibration

table is presented in Appendix A.

Hyperparameters. We discuss and report hyperparameter choices for all our experiments in

the Appendix A.

Figure 3 reports the numerical solution and a simulation for the Huggett

Numerical Results.

economy.

Log Aggregate Income z

Consumption c

2.00

1.75

1.50

1.25

1.00

0.75

0.50

Wealth b

0.02

0.00

0.02

0.04

0.0275

0.0250

0.0225

Time

0.0175

Time

Time

1.5e-5

100

Trajectory 1

Trajectory 2

Trajectory 3

Total assets B

1.2e-5

0.03

1.0e-5

7.5e-6

5.0e-6

2.5e-6

0.0e0

0.6

(d) Interest rate

z = 0.97, r = 0.031

z = 0.99, r = 0.027

z = 1.00, r = 0.024

z = 1.01, r = 0.021

z = 1.03, r = 0.017

z = 1.05, r = 0.017

Total assets B

0.04

100

1.00

(c) Aggregate consumption

0.01

1.02

100

0.02

0.0200

1.04

(b) Aggregate state

0.05

0.0300

Interest rate r

Interest rate r

0.0325

0.06

Trajectory 1

Trajectory 2

Trajectory 3

Trajectory 1

Trajectory 2

Trajectory 3

1.06

0.98

(a) Optimal consumption policy

0.0350

Trajectory 1

Trajectory 2

Trajectory 3

0.04

Aggregate consumption C

y = 0.55, z = 0.96, r = 0.033

y = 0.55, z = 1.04, r = 0.017

y = 1.82, z = 0.96, r = 0.033

y = 1.82, z = 1.04, r = 0.017

2.25

Aggregate saving S

2.50

0.4

0.2

0.0

0.2

0.4

Aggregate saving S(r, z)

0.6

0.8

-2.5e-6

(e) Saving schedule

Time

100

(f) Aggregate saving

Figure 3: Simulation Results

Panel (a) plots the optimal consumption policy as a function of wealth b on the horizontal

axis. Each line corresponds to one combination of individual income, aggregate income, and

prices. We choose the realizations of (yt , zt , pt ) that occur frequently in our simulations. The

policy is monotonically increasing and concave in b, as expected from standard theory: richer

households consume more, but at a decreasing marginal propensity.

Panels (b)-(d) display simulated time series for the aggregate state, aggregate consumption

and equilibrium prices. Panel (b) shows the exogenous process for log aggregate income zt .

Panel (c) plots aggregate consumption Ct , which co-moves with zt but is somewhat smoother

due to precautionary savings and imperfect risk sharing. Panel (d) shows the resulting interest

rate rt , which adjusts endogenously to clear the bond market in the presence of incomplete

markets and zero net bond supply.

Panel (e) revisits the equilibrium bond demand schedule S( p, z) discussed in Section 3.4,

now evaluated at the trained policy. For different realizations of the aggregate state z, the figure

shows how the aggregate demand for bonds varies with the interest rate. Market clearing

corresponds to the intersection of S( p, z) with zero.

Finally, Panel (f) plots the bond-market clearing residual along the simulated path, i.e. the

difference between aggregate bond holdings implied by households’ policies and the fixed

supply of zero. The residual remains very close to zero throughout the simulation. The small

deviations that do arise are due to numerical interpolation in prices rather than to a failure

of the algorithm to enforce equilibrium. In practice, these deviations are negligible both in

absolute terms. The average gap in bond market clearing for a single run is about 4.4 × 10−6 .

Partial equilibrium problem. To gauge the accuracy of our SRL approach, we first consider

a partial equilibrium (PE) version of the Huggett economy. In PE, households take as given

an exogenous Markov process for interest rates and solve their individual dynamic program

using either our SRL method or a standard value function iteration (VFI) algorithm. Because

the PE environment dispenses with the fixed point over prices and distributions, it is a setting

in which there is broad agreement on the correctness of conventional VFI solutions. This makes

it a natural benchmark against which to compare the policies implied by our method. We

describe the details of the PE specification and calibration in Appendix A.1.

y = 0.55, r = 0.019, z = 0.94

2.25

2.00

2.00

1.75

1.50

1.25

1.00

0.75

0.50

1.75

1.50

1.25

1.00

0.75

Wealth b

0.50

y = 1.82, r = 0.019, z = 0.94

Consumption c

1.8

1.6

1.4

Wealth b

VFI

Our Method

2.2

2.0

y = 1.82, r = 0.032, z = 1.06

2.4

VFI

Our Method

2.2

Consumption c

VFI

Our Method

2.25

Consumption c

Consumption c

y = 0.55, r = 0.032, z = 1.06

VFI

Our Method

2.0

1.8

1.6

1.4

1.2

1.2

Wealth b

Wealth b

Figure 4: Solution comparison for the PE problem: SRL vs VFI

Figure 4 reports the comparison. Each panel plots the optimal consumption policy as a

function of wealth b for four combinations of individual income y and interest rates r. The

dashed line shows the policy obtained from our SRL algorithm, while the solid line shows

the corresponding VFI solution in the PE environment. Across all four panels, the two sets

of policy functions are almost indistinguishable. This comparison is a first reassuring test of

the accuracy of our SRL method. It shows that, in a setting where a trusted VFI benchmark is

available, our SRL approach replicates the rational expectation solution closely.

Solutions with Lagged Price History.

A complementary way to assess the restrictiveness of

conditioning only on pt is to enlarge the observable state with lagged prices. Conceptually,

this moves us part of the way toward the full MA(∞) representation of the agent’s problem,

which in the limit would reproduce the rational expectations solution. Concretely, we now

re-solve the Huggett model allowing households to keep track of one lagged price, so that pt−1

becomes an additional state variable.

s2 = 0.55, zt = 0.982, pt = 0.029

Only current price

pt 1 = 0.027

pt 1 = 0.029

pt 1 = 0.031

1.75

2.00

Consumption c

Consumption c

2.00

1.50

1.25

1.00

1.75

1.00

0.50

0.50

Wealth s1

1.0

Wealth s1

1.4

Wealth s1

Wealth s1

1.8

1.6

1.4

Wealth s1

s2 = 1.82, zt = 1.019, pt = 0.020

2.4

Only current price

pt 1 = 0.018

pt 1 = 0.020

pt 1 = 0.022

2.2

1.4

1.0

2.0

1.6

1.0

Only current price

pt 1 = 0.018

pt 1 = 0.020

pt 1 = 0.022

2.2

1.8

1.2

Wealth s1

s2 = 1.00, zt = 1.019, pt = 0.020

Only current price

pt 1 = 0.023

pt 1 = 0.025

pt 1 = 0.027

2.0

1.2

0.50

Consumption c

Consumption c

1.6

s2 = 1.82, zt = 0.998, pt = 0.025

2.2

1.8

1.0

2.4

Only current price

pt 1 = 0.027

pt 1 = 0.029

pt 1 = 0.031

2.0

1.2

s2 = 1.82, zt = 0.982, pt = 0.029

2.2

Wealth s1

1.4

1.0

1.6

1.2

1.8

1.2

1.00

Consumption c

Consumption c

Consumption c

1.4

2.4

Consumption c

2.0

1.6

1.25

Only current price

pt 1 = 0.023

pt 1 = 0.025

pt 1 = 0.027

2.2

1.8

1.50

s2 = 1.00, zt = 0.998, pt = 0.025

Only current price

pt 1 = 0.027

pt 1 = 0.029

pt 1 = 0.031

2.0

1.75

0.75

s2 = 1.00, zt = 0.982, pt = 0.029

2.2

2.00

1.25

0.75

s2 = 0.55, zt = 1.019, pt = 0.020

Only current price

pt 1 = 0.018

pt 1 = 0.020

pt 1 = 0.022

2.25

1.50

0.75

s2 = 0.55, zt = 0.998, pt = 0.025

Only current price

pt 1 = 0.023

pt 1 = 0.025

pt 1 = 0.027

2.25

Consumption c

2.25

2.0

1.8

1.6

1.4

1.2

Wealth s1

Wealth s1

Figure 5: Consumption Policy Function with Price Lags pt−1 as State Variable

From a computational standpoint, this extension is straightforward: our method can accommodate a small number of lagged observables without reintroducing the curse of dimensionality. From an economic standpoint, it should, in principle, help agents forecast: because

prices are not Markov in pt alone, a longer price history ought to contain incremental information about future prices.

Figure 5 shows that, in practice, the effect of this additional information is minimal. The

figure compares solutions to the Huggett model when agents condition (i) only on the current

price pt (solid blue line) and (ii) on both pt and its lag pt−1 (dashed lines). Each panel plots

the consumption policy across wealth b for fixed values of the individual income state y, the

aggregate income state z, and the current interest rate r. Different dashed lines correspond to

different realizations of past prices.

Across all panels, the dashed lines lie almost on top of the solid line. That is, once we

fix the current state (yt , zt , pt ), optimal consumption is almost insensitive to the additional

information contained in pt−1 . This suggests that, at least in the Huggett environment, current

prices already summarize the relevant aspects of the history for household decisions, and that

extending the observable state to include one lag has only a negligible impact on behavior.

Next, we study how the quality

Dependence on the number of trajectories (sample size).

and stability of the learned policy depend on the number of simulated trajectories used for

training. Figure 6 summarizes these results.

Panel (a) reports the consumption policy obtained from a single training run with 512 simulated trajectories. This number of trajectories is a key hyperparameter in our algorithm: it

controls how many distinct state-price paths agents observe and learn from. The resulting

policy is monotone and concave in wealth, as theory would suggest.

2.50

2.25

2.00

1.75

1.50

1.25

1.75

1.50

1.25

1.00

1.00

0.75

0.75

0.50

2.50

Wealth b

(a) Policy with Nsample = 512

0.50

2.00

1.50

1.25

0.75

Wealth b

(c) Policy CI with Nsample = 2048

1.25

0.75

1.50

1.00

1.75

1.00

y = 0.55, z = 0.96, r = 0.034

y = 0.55, z = 1.04, r = 0.016

y = 1.82, z = 0.96, r = 0.034

y = 1.82, z = 1.04, r = 0.016

2.25

1.75

0.50

2.50

Consumption c

2.00

Wealth b

(b) Policy CI with Nsample = 512

y = 0.55, z = 0.96, r = 0.033

y = 0.55, z = 1.04, r = 0.019

y = 1.82, z = 0.96, r = 0.033

y = 1.82, z = 1.04, r = 0.019

2.25

Consumption c

y = 0.55, z = 0.96, r = 0.033

y = 0.55, z = 1.04, r = 0.017

y = 1.82, z = 0.96, r = 0.033

y = 1.82, z = 1.04, r = 0.017

2.25

Consumption c

2.00

Consumption c

2.50

y = 0.55, z = 0.96, r = 0.033

y = 0.55, z = 1.04, r = 0.017

y = 1.82, z = 0.96, r = 0.033

y = 1.82, z = 1.04, r = 0.017

0.50

Wealth b

(d) Policy CI with Nsample = 24

Figure 6: Dependence of Policies on Sample Size in the Huggett model

Panel (b) turns to sampling uncertainty. Here we compute pointwise 95\% confidence intervals (CIs) for the policy based on 10 independent training runs, each again using 512 trajectories. The figure shows that the confidence bands are quite tight across the wealth distribution.

The bands widen modestly at higher wealth levels. This pattern is natural: states with high

wealth are visited only rarely in the simulations — indeed, beyond roughly b > 10 there is essentially no mass in the stationary distribution, so the algorithm has fewer observations from

which to learn. Regions of the state space that are visited infrequently thus come with more

sampling noise in the estimated policy.

Panels (c) and (d) vary the number of training trajectories to make this dependence more

transparent. In Panel (c), we increase the number of trajectories. The point estimates of the

policy change very little, but the confidence bands shrink, including in the high-wealth region.

In other words, feeding the algorithm more data primarily reduces uncertainty; it does not

systematically move the policy itself. Conversely, Panel (d) shows the case with fewer trajectories. Here the confidence intervals become much wider, especially at high wealth levels

where visits are sparse. Put differently, when agents have learned from only a small number

of simulations, agents who happen to find themselves in the same high-wealth state can end

up taking quite different actions across independent runs.

These results suggest a useful way to think about our stochastic solution method. If we

could associate to each point in the state space a simple measure of “confidence” — for example, based on the width of the confidence band or on how often that state is visited in simulation — it would reveal that agents are very certain about their behavior in frequently visited

states, but much less certain in rare states. Our experiments indicate that, for economically relevant regions of the state space (low and medium wealth), the learned policies are both stable

and precise, while the main residual uncertainty is confined to tails that households almost

never reach in practice.

\subsection{Krusell-Smith Model}

Setup. The household side of the Krusell-Smith economy is as in the Huggett model, except that financial wealth is now productive capital owned by households and rented

to a representative firm. The firm uses capital and labor to produce according to

Yt = zt Ktα L1t −α .

Under perfect competition, factor prices equal marginal products, wt = (1 − α) YLtt and rtK = α KYtt ,

where wt is the real wage rate and rtK the rental rate of capital. Since households own the capital

and pay for depreciation, their net rate of return on capital is rt = rtK − δ. The market clearing

condition for capital is

Z

b dGt (b, y) = Kt ,

and labor market clearing condition is given by Lt = 1 because each household supplies one

unit of labor inelastically.

Calibration. One period corresponds to a year. On the preference side, we set β = 0.95 and

use CRRA preferences with coefficient of relative risk aversion σ = 3. On the production side,

we set the capital share to α = 0.36 and the depreciation rate to δ = 0.08, standard in the

quantitative macro literature. For idiosyncratic income, we retain the same AR(1) specification

and parameters as in the Huggett model, so that the cross-sectional heterogeneity is directly

comparable across the two experiments. Aggregate productivity zt also follows a log AR(1)

process with persistence ρz = 0.9 and innovation volatility νz = 0.03. All log AR(1) processes

are discretized on finite grids using a standard Tauchen procedure; the details of the grids are

reported in Appendix A.

Figure 7 plots simulation results for our solution of the Krusell-Smith

Numerical Results.

model. We start in Panel (a) with the exogenous aggregate productivity process zt , and plot in

Panel (b) aggregate capital Kt , in Panel (c) aggregate consumption Ct , in Panel (d) the aggregate

rental rate rtK and in Panel (e) the aggregate wage wt . As in standard neoclassical models, Kt

and Ct comove strongly with zt , while rtK and wt move in opposite directions.

6.5

Aggregate capital k

log TFP(z)

0.1

0.0

0.1

1.8

Trajectory 1

Trajectory 2

Trajectory 3

7.0

Aggregate consumption C

Trajectory 1

Trajectory 2

Trajectory 3

0.2

6.0

5.5

5.0

4.5

4.0

0.2

3.5

100 125 150 175 200

Time

(a) log TFP log(z)

100

Time

125

150

175

200

1.3

Wage w

0.05

1.2

1.1

0.04

1.0

0.03

0.9

100

Time

125

150

175

200

(d) Rental rate r

1.4

1.3

1.2

1.1

1.8

1.4

0.06

1.5

100

Time

125

150

175

100

Time

125

150

175

200

(c) Aggregate consumption C

Trajectory 1

Trajectory 2

Trajectory 3

1.5

0.07

Rental rate r

(b) Aggregate capital K

Trajectory 1

Trajectory 2

Trajectory 3

0.08

1.6

Aggregate consumption C

Trajectory 1

Trajectory 2

Trajectory 3

1.7

200

(e) Wage w

All trajectories

1.7

1.6

1.5

1.4

1.3

1.2

1.1

0.2

0.1

0.0

log TFP z

0.1

0.2

(f) C vs log TFP(z)

Figure 7: Simulation Results

The final panel (f) presents a scatter plot of aggregate consumption Ct against the exogenous productivity shock zt . We find substantial vertical dispersion: for a given realization of

zt , different periods in the simulation exhibit quite different levels of aggregate consumption.

If all points lay on a single curve, then the same aggregate productivity level would always be

associated with the same Ct . Instead, the cross-sectional wealth distribution shifts over time in

ways that matter for aggregates, so the mapping zt 7→ Ct is history-dependent. The blue dots

in Panel (f) represent states actually visited along the simulated path, so this vertical spread

measures the quantitative importance of distributional dynamics for aggregate outcomes. For

values of zt very close to zero, simulated realizations of Ct range roughly from 1.24 to 1.6, i.e.

a difference on the order of 20\% of steady-state consumption. This illustrates that, even in this

relatively simple heterogeneous-agent model, the cross-sectional distribution has a non-trivial

impact on the aggregate response.

Dependence on the number of trajectories (sample size). We now illustrate how the learned

policy depends on the number of simulated trajectories used for training, similar to our discussion of Figure 6 for the Huggett model. Panels (a) and (b) report the consumption policy

3.0

2.0

1.5

1.0

0.5

Wealth b

(a) Policy with Nsample = 512

1.5

0.5

3.0

y = 0.55, r = 0.049, w = 1.14

y = 1.00, r = 0.049, w = 1.14

y = 1.82, r = 0.049, w = 1.14

2.0

1.5

1.0

Wealth b

(b) Policy CI with Nsample = 512

y = 0.55, r = 0.049, w = 1.14

y = 1.00, r = 0.049, w = 1.14

y = 1.82, r = 0.049, w = 1.14

2.5

Consumption c

2.5

Consumption c

2.0

1.0

3.0

0.5

y = 0.55, r = 0.049, w = 1.14

y = 1.00, r = 0.049, w = 1.14

y = 1.82, r = 0.049, w = 1.14

2.5

Consumption c

2.5

Consumption c

3.0

y = 0.55, r = 0.049, w = 1.14

y = 1.00, r = 0.049, w = 1.14

y = 1.82, r = 0.049, w = 1.14

2.0

1.5

1.0

Wealth b

(c) Policy CI with Nsample = 2048

0.5

Wealth b

(d) Policy CI with Nsample = 24

Figure 8: Dependence of Policies on Sample Size in the Krusell and Smith (1998) model

from a single training run and the associated 95\% confidence intervals based on 10 independent training runs, both using 512 trajectories, respectively. The resulting policy is monotone

and concave in wealth and visually very similar to the policies we documented in the Huggett

experiment.There is only mild sampling uncertainty. Confidence bands are tight, especially

for low wealth levels.

Panel (c) increases the number of trajectories to 2048 and illustrates that confidence bands

shrink even for the highest wealth levels. Panel (d) instead reduces the number of trajectories

to 24 and illustrates that sampling uncertainty across runs increases substantially, especially at

high wealth levels which are visited more rarely during the simulation.

Comparison to rational expectation solution.

The partial equilibrium comparison we used

in Section 4.1 isolates the individual dynamic programming problem, where there is broad

agreement on the accuracy of conventional VFI solutions. In general equilibrium, by contrast,

obtaining the rational expectations (RE) solution requires treating the entire cross-sectional

distribution as a state variable and solving the Master equation, a problem of much higher

computational complexity. A growing literature proposes global solution methods for such

RE equilibria. One recent example is the DeepHAM approach of Han et al. (2021), which uses

deep neural networks to approximate high-dimensional policy and value functions.

To benchmark our SRL method in this environment, we compare it directly to DeepHAM.

Because the RE policy functions in DeepHAM conditions on the full cross-sectional distribution, whereas our approach conditions only on prices, the policy functions are not directly

comparable. We therefore focus on equilibrium dynamics. Specifically, we initialize both

economies from the same cross-sectional distribution and expose them to identical sequences

of aggregate shocks. Figure 9 reports the resulting paths of aggregate consumption and capital.

Across both panels, the two methods generate nearly indistinguishable aggregate dynamics.

This comparison indicates that, at least in the Krusell-Smith environment, our SRL approach can replicate the rational expectations solution while retaining the flexibility and scalability of a reinforcement-learning implementation.

DeepHAM

Our Method

41

2.90

Aggregate capital K

Aggregate consumption C

DeepHAM

Our Method

2.95

2.85

2.80

2.75

200

400

Time

600

800

1000

200

400

Time

600

800

1000

Figure 9: Comparison to RE Solution with the DeepHAM method in Han et al. (2021)

4.3

HANK Model

Our final benchmark economy is a one-account HANK model with sticky prices. This environment adds nominal rigidities and a richer firm block to the incomplete-markets structure

above.

Setup. The problem of household i is identical to that in the Huggett model, except that labor supply

is now endogenous. In sequence form,

∞

vi,0 = max E0 ∑ βt u(ci,t , ni,t )

\{ci,t ,ni,t \}

s.t.

t =0

ci,t + bi,t+1 = (1 + rt )bi,t + wt yi,t ni,t + dt − Tt ,

bi,t+1 ≥ 0,

where wt is the real wage, dt denotes dividend payouts, and Tt is a lump-sum tax. Households

are the ultimate owners of firms but equity shares are not traded. The idiosyncratic income

process yi,t is as in the Huggett model.

On the production side, a competitive final-good firm aggregates a continuum of intermediate inputs using a CES production technology with elasticity of substitution ε. Denoting by

Yt aggregate output of the final good, cost minimization implies that demand for intermediate

input j is



y j,t =

Pj,t

Pt

−ε

Yt ,

(22)

where Pj,t is the price of good j and

Pt =

Z 1

1− ε

Pj,t

dj

 1−1 ε

is the aggregate price index.

Each intermediate good j is produced by a monopolistically competitive firm with technology y j,t = zt L j,t . The productivity term zt is common to all firms and follows a Markov process

zt+1 ∼ Tz (· | zt ). Firm j chooses its price \{ Pj,t \} to maximize the discounted value of profits

subject to a quadratic adjustment cost as in Rotemberg (1982):

∞

Jj,0 = max E0 ∑

\{ Pj,t \}

1

R0−→

t

t =0



Pj,t

wt

θ

y j,t − y j,t −

Pt

zt



Pj,t − Pj,t−1

Pj,t−1

2 

Yt

subject to the demand function (22) and taking as given an initial price Pj,−1 . Here R0→t =

R1 × . . . × Rt denotes the gross real interest rate between periods 0 and t with Rt = 1 + rt .

We focus on a symmetric distribution of initial prices, Pj,−1 = Pj0 ,−1 , which implies symmetry ex post. In equilibrium we therefore have Pj,t = Pt and y j,t = Yt for all j. Denoting net

inflation by

Πt =

Pt − Pt−1

,

Pt−1

the firm’s problem gives rise to the New Keynesian Phillips curve

Π t (1 + Π t ) =

ε

θ



wt

ε−1

−

zt

ε







1 Yt+1

+ Et R −

Π

(

+

Π

)

t +1

t +1 .

t +1

Yt

(23)

The first term captures the gap between real marginal cost wztt and the desired markup ε−ε 1 ,

while the second term reflects expected future inflation, discounted by the real interest rate

and scaled by output growth. Given inflation, firm dividend payments are



dt =



wt

θ

1−

Yt − Π2t Yt .

zt

Monetary policy follows a Taylor rule

1 + it+1 = R̄(1 + Πt )φ eet ,

where φ > 1 and the monetary policy shock et+1 ∼ Te (· | et ) follows a Markov process. The

real and nominal interest rates are linked by the Fisher equation

Rt =

1 + it

.

1 + Πt

The government has a fixed supply of debt B outstanding and finances interest payments using

lump-sum taxes on households, rt B = Tt .

Finally, three markets must clear in equilibrium. Goods market clearing requires

Yt =

Z

θ

c(b, y)dG (b, y) + Π2t Yt .

so aggregate output is absorbed by consumption and price-adjustment costs. Labor market

clearing implies

Lt =

Z

and the bond market clears when

B=

n(b, y)dG (b, y),

Z

b dG (b, y),

where we assume that bonds are in fixed positive supply B. The definition of competitive

equilibrium is standard.

Firm Policy Gradient Method for the Phillips Curve. The price-setting block introduces an

additional difficulty relative to the Huggett and Krusell-Smith models: Firm optimality gives

rise to the forward-looking Phillips curve (23). Standard approaches typically parameterize the

conditional expectation term in Equation (23) and solve a non-trivial fixed point for the law of

motion of inflation; see for example Kase et al. (2024); Fernández-Villaverde et al. (2024b).

By contrast, we treat the firm problem exactly as we treat the household problem and solve

it using the same SPG method. In practice, this means that we represent the firm’s inflation

policy as a function of individual firm states, aggregate shocks, and prices that are payoff

relevant: Πt = Π(wt , zt ). All policy functions are updated simultaneously. Households and

firms thus learn jointly from the same simulated trajectories, and there is no separate fixedpoint step for equilibrium expectations.

A detailed description of the implementation is provided in Appendix A.3. Here we simply

note that, in our experiments, this symmetric treatment of households and firms has very good

convergence properties. As Table 1 shows, solving the HANK model with the forward-looking

Phillips curve is only mildly more costly than solving the baseline Huggett model, despite the

added forward-looking structure in the firm block.

Calibration. A time period corresponds to a year, with β = 0.95. Preferences are CRRA and

separable in consumption and labor, u(c, n) = 1−1 σ c1−σ − 1+1 η n1+η , with coefficient of relative

risk aversion σ = 1 implying log utility over consumption, and inverse Frisch elasticity η = 1.

On the production side, we set the elasticity of substitution across intermediate goods to

ε = 10 and the Rotemberg adjustment cost parameter to θ = 100. We set the Taylor rule

coefficient to φ = 1.5, and we set the fixed government bond supply to B = 5.

This economy features three shocks: one idiosyncratic and two aggregate. For households’

idiosyncratic income process, we use the same AR(1) specification and parameterization as in

the Huggett model so that cross-sectional heterogeneity is directly comparable across applications. Aggregate risk comprises the TFP process zt and the monetary policy shock et , both

following AR(1) processes. We set their persistence to ρz = ρe = 0.9 and their innovation

volatilities to νz = 0.07 and νe = 0.002, respectively. We use a standard Tauchen procedure to

discretize all three processes on finite grids and report all remaining details in Appendix A.

Numerical Results.

Figure 10 reports the optimal policy functions for households and firms

in the HANK model. Panels (a) and (b) show household consumption and labor supply policies, while Panel (c) displays the firm’s inflation policy. For each policy, we plot the point

estimate together with confidence bands obtained from 10 independent training runs with

N = 512 simulated trajectories each.

1.8

y = 0.55, r = 0.026, w = 0.88

y = 1.00, r = 0.026, w = 0.88

y = 1.82, r = 0.026, w = 0.88

1.4

1.4

1.2

1.0

1.2

1.0

0.8

0.8

0.6

0.6

0.4

y = 0.55, r = 0.026, w = 0.88

y = 1.00, r = 0.026, w = 0.88

y = 1.82, r = 0.026, w = 0.88

1.6

Labor supply n

Consumption c

1.6

Wealth b

(a) Consumption policy

Wealth b

(b) Labor supply policy

w = 0.87

w = 0.89

w = 0.91

0.02

Inflation

0.01

0.00

0.01

0.02

0.2

0.1

0.0

log TFP log(z)

0.1

0.2

(c) Inflation policy

Figure 10: Household and Firm Policy Functions in HANK

Panel (a) shows that consumption is increasing and concave in wealth and increasing in

idiosyncratic labor productivity, as standard theory would suggest. Panel (b) shows the corresponding labor supply policy, which is decreasing and convex in wealth and increasing in

labor productivity: richer households work less at the margin, while high-productivity households supply more labor.

Panel (c) displays the firm’s inflation policy function. It is decreasing and almost linear in

log aggregate productivity and increasing in marginal cost. In particular, positive supply (TFP)

shocks lower marginal cost and induce lower inflation, consistent with the New Keynesian

Phillips curve (23).

Across all three panels, the confidence bands are tight over the bulk of the wealth distribution and widen somewhat only in the far tails, where states are rarely visited in simulation.

The firm’s policy is the most challenging to learn — reflected in somewhat wider confidence

bands — but remains well behaved and economically sensible. Overall, Figure 10 suggests

that our SPG method recovers accurate policy functions in this richer HANK environment.

Figure 11 reports simulated time series for the HANK model. Panels (a) and (b) display the

two aggregate shocks: log TFP zt and the monetary policy shock et . Panels (c)-(f) then show

the induced time series for the real interest rate, aggregate consumption, aggregate savings,

and inflation.

0.20

Trajectory 1

Trajectory 2

0.00

0.05

Interest Rate r

Monetary shock e

0.05

0.10

0.002

0.000

0.002

0.15

Time

100

Trajectory 1

Trajectory 2

Time

0.95

0.90

100

(d) Aggregate consumption C

Time

100

Trajectory 1

Trajectory 2

0.02

0.01

5.02

5.00

4.98

0.00

0.01

4.96

0.02

4.92

Time

(c) Interest rate r

4.94

Inflation

Aggregate saving S

1.00

0.02

100

Trajectory 1

Trajectory 2

Total assets B

5.06

5.04

0.03

(b) Monetary shock e

1.05

0.04

0.01

(a) log TFP log(z)

Aggregate consumption C

Trajectory 1

Trajectory 2

0.05

0.004

0.10

log TFP log(z)

0.006

Trajectory 1

Trajectory 2

0.15

Time

100

(e) Aggregate saving S

Time

100

(f) Inflation Π

Figure 11: Simulated Trajectories in HANK

Aggregate consumption in Panel (d) moves procyclically with productivity: positive TFP

shocks raise Yt and, through higher labor income and lower marginal costs, also increase Ct .

Inflation in Panel (f) is countercyclical with respect to TFP, consistent with the Phillips curve

(23): favorable supply shocks reduce marginal costs and put downward pressure on inflation.

In our calibration macro aggregates appear more sensitive to TFP innovations than to monetary shocks.

Panel (e) plots households’ aggregate demand for bonds, together with the fixed supply

B = 5. Asset demand fluctuates tightly around the supply level, and deviations from the

horizontal supply line measure residuals in the bond market clearing condition. On average,

the relative absolute deviation of aggregate demand from supply is 0.22\%. These residuals

primarily reflect the use of linear interpolation in solving for market-clearing prices.

\section{Conclusion}

We develop a new structural reinforcement learning (SRL) approach to formulating and globally solving heterogeneous agent models with aggregate risk. We replace the cross-sectional

distribution with low-dimensional prices as state variables and let agents compute price expectations directly from simulated paths. Our approach differs from standard RL in that we

assume that agents have structural knowledge about the dynamics of their own individual

states (e.g. their budget constraint and idiosyncratic income process). Our structural policy

gradient (SPG) algorithm sidesteps the Master equation and efficiently handles heterogeneous

agent models traditional methods struggle with, like those with nontrivial market-clearing

conditions. By imposing that policy functions depend only on current prices (or a short price

history) we keep the state space low-dimensional so that we can work with a grid-based (tabular) approach rather than deep neural networks.

We implement our method in JAX and conduct computational experiments for three benchmark models in macroeconomics – the Huggett (1993) model, the Krusell and Smith (1998)

model, and a one-asset HANK model with sticky prices. In all three cases, the SPG algorithm

converges in only a few minutes. In the Krusell-Smith model, the resulting policies are close

to those obtained from alternative global solutions of the rational expectations equilibrium.

Allowing agents to condition on a longer history of lagged prices hardly moves the solution,

indicating that much of the relevant information for forecasting future prices is already contained in current prices. In the HANK model, we show how the same approach can be used

symmetrically on the household and firm sides to globally solve the forward-looking New

Keynesian Phillips curve.

The core idea of our approach – that agents form price expectations by sampling – could, in

principle, serve as a building block for an empirically realistic theory of expectations formation

in macroeconomics (which the algorithm in this paper is not). The plausibility of RL-based approaches is supported by evidence that reinforcement learning underpins a substantial share

of human and animal learning.26 To develop such an RL-based theory of expectations formation would require several modifications to our SRL method. First, the algorithm would need

to be converted to a fully online, incremental RL algorithm, with agents updating policies

and value estimates continuously while interacting with their environment, rather than only

after observing N price trajectories of length T. Second, the assumption that agents sample

equilibrium prices in an unbiased way would likely need to be relaxed. Empirical evidence

suggests that people disproportionately weight certain personal experiences (e.g. Malmendier

and Nagel, 2011, 2015), so incorporating biased or experience-weighted sampling would be

more realistic. Third, our SRL agents form price expectations in a model-free manner; this

may be too simplistic and one could instead treat prices using a model-based RL approach.

References

Achdou, Yves, Jiequn Han, Jean-Michel Lasry, Pierre-Louis Lions, and Benjamin Moll, “Income and Wealth Distribution in Macroeconomics: A Continuous-Time Approach,” The Review of Economic Studies, 04 2021, 89 (1), 45–86.

Ahn, SeHyoun, Greg Kaplan, Benjamin Moll, Thomas Winberry, and Christian Wolf,

“When Inequality Matters for Macro and Macro Matters for Inequality,” NBER Macroeconomics Annual, 2018, 32 (1), 1–75.

26 See, for example, Niv (2009), Glimcher (2011), Caplin and Dean (2008), Glimcher et al. (2013), Gershman and

Daw (2017), and Barberis and Jin (2023).

Aiyagari, S. Rao, “Uninsured Idiosyncratic Risk and Aggregate Saving,” The Quarterly Journal

of Economics, August 1994, 109 (3), 659–84.

Auclert, Adrien, Bence Bardóczy, Matthew Rognlie, and Ludwig Straub, “Using the

Sequence-Space Jacobian to Solve and Estimate Heterogeneous-Agent Models,” Econometrica, September 2021, 89 (5), 2375–2408.

, Matthew Rognlie, and Ludwig Straub, “Fiscal and Monetary Policy with Heterogeneous

Agents,” Annual Review of Economics, 2025, 17 (Volume 17, 2025), 539–562.

, Rodolfo Rigato, Matthew Rognlie, and Ludwig Straub, “Beyond Certainty Equivalence:

Second-order Dynamics in the Sequence Space,” Presentation slides 2025.

Azinovic, Marlon, Luca Gaegauf, and Simon Scheidegger, “Deep Equilibrium Nets,” International Economic Review, 2022, 63 (4), 1471–1525.

Azinovic-Yang, Marlon and Jan Žemlička, “Deep Learning in the Sequence Space,” arXiv

preprint arXiv:2509.13623, 2025.

Barberis, Nicholas and Lawrence Jin, “Model-free and Model-based Learning as Joint Drivers

of Investor Behavior,” NBER Working Papers 31081 March 2023.

Bilal, Adrien, “Solving Heterogeneous Agent Models with the Master Equation,” NBER Working Papers 31103, National Bureau of Economic Research 2023.

Bradbury, James, Roy Frostig, Peter Hawkins, Matthew James Johnson, Chris Leary, Dougal

Maclaurin, George Necula, Adam Paszke, Jake VanderPlas, Skye Wanderman-Milne, and

Qiao Zhang, “JAX: composable transformations of Python+NumPy programs,” 2018.

Branch, William A., “Restricted Perceptions Equilibria and Learning in Macroeconomics,”

Post Walrasian Macroeconomics: Beyond the Dynamic Stochastic General Equilibrium Model. Cambridge University Press, New York, 2006, pp. 135–160.

Bray, Margaret, “Learning, Estimation, and the Stability of Rational Expectations,” Journal of

Economic Theory, April 1982, 26 (2), 318–339.

Calvano, Emilio, Giacomo Calzolari, Vincenzo Denicolo, and Sergio Pastorello, “Artificial

intelligence, algorithmic pricing, and collusion,” American Economic Review, 2020, 110 (10),

3267–3297.

Caplin, Andrew and Mark Dean, “Dopamine, Reward Prediction Error, and Economics,” The

Quarterly Journal of Economics, 2008, 123 (2), 663–701.

Cardaliaguet, Pierre, François Delarue, Jean-Michel Lasry, and Pierre-Louis Lions, The Master Equation and the Convergence Problem in Mean Field Games, Vol. 201 of Annals of Mathematics

Studies, Princeton University Press, 2019.

Chen, Mingli, Andreas Joseph, Michael Kumhof, Xinlei Pan, and Xuan Zhou, “Deep Reinforcement Learning in a Monetary Model,” 2023.

Cho, In-Koo and Thomas J. Sargent, Self-confirming Equilibria, Palgrave Macmillan UK, December 2016.

de la Barrera, Marc and Tim de Silva, “Model-Agnostic Dynamic Programming,” Working

Paper, Stanford University 2024.

DeepSeek-AI, “DeepSeek-R1: Incentivizing Reasoning Capability in LLMs via Reinforcement

Learning,” 2025.

Den Haan, Wouter J., “Heterogeneity, Aggregate Uncertainty, and the Short-Term Interest

Rate,” Journal of Business \& Economic Statistics, October 1996, 14 (4), 399–411.

, Kenneth L. Judd, and Michel Juillard, “Computational suite of models with heterogeneous agents: Incomplete markets and aggregate uncertainty,” Journal of Economic Dynamics

and Control, 2010, 34 (1), 1–3.

Dou, Winston Wei, Itay Goldstein, and Yan Ji, “AI-powered trading, algorithmic collusion,

and price efficiency,” 2025.

Duarte, Victor, Diogo Duarte, and Dejanir H Silva, “Machine Learning for Continuous-Time

Finance,” The Review of Financial Studies, 2024, 37 (11), 3217–3271.

Erev, Ido and Alvin E. Roth, “Learning in Extensive-Form Games: Experimental Data and

Simple Dynamic Models in the Intermediate Term,” Games and Economic Behavior, 1995, 8

(1), 164–212.

and , “Predicting How People Play Games: Reinforcement Learning in Experimental

Games with Unique, Mixed Strategy Equilibria,” American Economic Review, 1998, pp. 848–

881.

Evans, George W. and Seppo Honkapohja, Learning and Expectations in Macroeconomics,

Princeton University Press, 2001.

Farmer, Leland E. and Alexis Akira Toda, “Discretizing nonlinear, non-Gaussian Markov processes with exact conditional moments,” Quantitative Economics, 2017, 8 (2), 651–683.

Favilukis, Jack, Sydney C. Ludvigson, and Stijn Van Nieuwerburgh, “The Macroeconomic

Effects of Housing Wealth, Housing Finance, and Limited Risk Sharing in General Equilibrium,” Journal of Political Economy, 2017, 125 (1), 140–223.

Fernández-Villaverde, Jesús, Galo Nuño, and Jesse Perla, “Taming the Curse of Dimensionality: Quantitative Economics with Deep Learning,” NBER Working Papers 33117, National

Bureau of Economic Research November 2024.

, , and Samuel Hurtado, “Financial Frictions and the Wealth Distribution,” Econometrica,

May 2023, 91 (3), 869–901.

, , Joël Marbet, and Omar Rachedi, “Inequality and the Zero Lower Bound,” Journal of

Econometrics, 2024, p. 105819.

Freeman, C. Daniel, Erik Frey, Anton Raichuk, Sertan Girgin, Igor Mordatch, and Olivier

Bachem, “Brax – A Differentiable Physics Engine for Large Scale Rigid Body Simulation,”

2021.

Fudenberg, Drew and David K. Levine, “Whither Game Theory? Towards a Theory of Learning in Games,” Journal of Economic Perspectives, 2016, 30 (4), 151–170.

Gabriele, Federico, Aldo Glielmo, and Marco Taboga, “Heterogeneous RBCs via deep multiagent reinforcement learning,” 2025.

Gershman, Samuel J. and Nathaniel D. Daw, “Reinforcement Learning and Episodic Memory

in Humans and Animals: An Integrative Framework,” Annual Review of Psychology, 2017, 68

(Volume 68, 2017), 101–128.

Giusto, Andrea, “Adaptive Learning and Distributional Dynamics in an Incomplete Markets

Model,” Journal of Economic Dynamics and Control, 2014, 40, 317–333.

Glimcher, Paul W., “Understanding Dopamine and Reinforcement Learning: The Dopamine

Reward Prediction Error Hypothesis,” Proceedings of the National Academy of Sciences, 2011,

108 (supplement 3), 15647–15654.

, Colin Camerer, Ernst Fehr, and Russell Poldrack, Neuroeconomics: Decision Making and the

Brain Academic Press, Academic Press, 2013.

Gomes, Francisco and Alexander Michaelides, “Asset Pricing with Limited Risk Sharing and

Heterogeneous Agents,” Review of Financial Studies, January 2008, 21 (1), 415–448.

Gopalakrishna, Goutham, Zhouzhou Gu, and Jonathan Payne, “Asset Pricing, Participation

Constraints, and Inequality,” Working Paper 2024.

Gu, Zhouzhou, Mathieu Laurière, Sebastian Merkel, and Jonathan Payne, “Global Solutions

to Master Equations for Continuous Time Heterogeneous Agent Macroeconomic Models,”

arXiv preprint arXiv:2406.13726, 2024.

Guarda, Sebastian, “Narrow and Short Beliefs in Macroeconomics with Heterogeneous

Agents,” Technical Report, Princeton University November 2025. Working paper, available

at https://files.sebastianguarda.com/JMP.pdf.

Han, Jiequn, Yucheng Yang, and Weinan E, “DeepHAM: A Global Solution Method For Heterogeneous Agent Models With Aggregate Shocks,” arXiv preprint arXiv:2112.14377, 2021.

Hausknecht, Matthew and Peter Stone, “Deep Recurrent Q-Learning for Partially Observable

MDPs,” 2017.

Heathcote, Jonathan, Kjetil Storesletten, and Giovanni L. Violante, “Quantitative Macroeconomics with Heterogeneous Households,” Annual Review of Economics, 05 2009, 1 (1), 319–

354.

Hu, Yuanming, Luke Anderson, Tzu-Mao Li, Qi Sun, Nathan Carr, Jonathan Ragan-Kelley,

and Frédo Durand, “DiffTaichi: Differentiable Programming for Physical Simulation,” 2020.

Huang, Ji, “Breaking the Curse of Dimensionality in Heterogeneous-Agent Models: A Deep

Learning-Based Probabilistic Approach,” 2023.

Huggett, Mark, “The risk-free rate in heterogeneous-agent incomplete-insurance economies,”

Journal of Economic Dynamics and Control, 1993, 17 (5-6), 953–969.

Jacobson, Margaret M., “Beliefs, Aggregate Risk, and the U.S. Housing Boom,” Technical Report 2025.

Kahou, Mahdi Ebrahimi, Jesús Fernández-Villaverde, Jesse Perla, and Arnav Sood, “Exploiting symmetry in high-dimensional dynamic programming,” Technical Report 2021.

Kaplan, Greg, Kurt Mitman, and Giovanni L. Violante, “The Housing Boom and Bust: Model

Meets Evidence,” Journal of Political Economy, 2020, 128 (9), 3285–3345.

Kase, Hanno, Leonardo Melosi, and Matthias Rottner, “Estimating Nonlinear Heterogeneous

Agent Models with Neural Networks,” Working Paper, FRB Chicago 2024.

Krueger, D., K. Mitman, and F. Perri, “Chapter 11 - Macroeconomics and Household Heterogeneity,” in John B. Taylor and Harald Uhlig, eds., John B. Taylor and Harald Uhlig, eds., Vol. 2

of Handbook of Macroeconomics, Elsevier, 2016, pp. 843–921.

Krusell, Per and Anthony A. Smith, “Income and wealth heterogeneity, portfolio choice, and

equilibrium asset returns,” Macroeconomic dynamics, 1997, 1 (2), 387–422.

and , “Income and Wealth Heterogeneity in the Macroeconomy,” Journal of Political Economy, October 1998, 106 (5), 867–896.

and , Quantitative Macroeconomic Models with Heterogeneous Agents Econometric Society

Monographs, Cambridge University Press, 2006.

Lasry, Jean-Michel and Pierre-Louis Lions, “Mean field games,” Japanese Journal of Mathematics, 2007, 2, 229–260.

Laurière, Mathieu, Sarah Perrin, Julien Pérolat, Sertan Girgin, Paul Muller, Romuald Élie,

Matthieu Geist, and Olivier Pietquin, “Learning in Mean Field Games: A Survey,” 2024.

, , Sertan Girgin, Paul Muller, Ayush Jain, Theophile Cabannes, Georgios Piliouras,

Julien Perolat, Romuald Elie, Olivier Pietquin, and Matthieu Geist, “Scalable Deep Reinforcement Learning Algorithms for Mean Field Games,” in Kamalika Chaudhuri, Stefanie

Jegelka, Le Song, Csaba Szepesvari, Gang Niu, and Sivan Sabato, eds., Proceedings of the

39th International Conference on Machine Learning, Vol. 162 of Proceedings of Machine Learning

Research PMLR 17–23 Jul 2022, pp. 12078–12095.

Lee, Donghoon and Kenneth I. Wolpin, “Intersectoral Labor Mobility and the Growth of the

Service Sector,” Econometrica, 2006, 74 (1), 1–46.

Llull, Joan, “Immigration, Wages, and Education: A Labour Market Equilibrium Structural

Model,” The Review of Economic Studies, 2018, 85 (3), 1852–1896.

Maliar, Lilia and Serguei Maliar, “Deep learning: Solving HANC and HANK models in the

absence of Krusell-Smith aggregation,” Available at SSRN 3758315, 2020.

, , and Pablo Winant, “Deep Learning for Solving Dynamic Economic Models,” Journal of

Monetary Economics, 2021, 122 (C), 76–101.

Malmendier, Ulrike and Stefan Nagel, “Depression Babies: Do Macroeconomic Experiences

Affect Risk Taking?,” The Quarterly Journal of Economics, 2011, 126 (1), 373–416.

and , “Learning from Inflation Experiences,” The Quarterly Journal of Economics, October

2015, 131 (1), 53–87.

Marcet, Albert and Thomas J. Sargent, “Convergence of Least Squares Learning Mechanisms

in Self-referential Linear Stochastic Models,” Journal of Economic Theory, August 1989, 48 (2),

337–368.

Mnih, Volodymyr, Koray Kavukcuoglu, David Silver, Andrei A. Rusu, Joel Veness, Marc G.

Bellemare, Alex Graves, Martin Riedmiller, Andreas K. Fidjeland, Georg Ostrovski, Stig

Petersen, Charles Beattie, Amir Sadik, Ioannis Antonoglou, Helen King, Dharshan Kumaran, Daan Wierstra, Shane Legg, and Demis Hassabis, “Human-level control through

deep reinforcement learning,” Nature, Feb 2015, 518 (7540), 529–533.

Moll, Benjamin, “The Trouble with Rational Expectations in Heterogeneous Agent Models: A

Challenge for Macroeconomics,” 2025.

Murphy, Kevin, “Reinforcement Learning: An Overview,” 2025.

Niv, Yael, “Reinforcement Learning in the Brain,” Journal of Mathematical Psychology, 2009, 53

(3), 139–154. Special Issue: Dynamic Decision Making.

OpenAI, “OpenAI o1 System Card,” 2024.

Payne, Jonathan, Adam Rebei, and Yucheng Yang, “Deep Learning for Search and Matching

Models,” 2024.

Quadrini, Vincenzo and José-Vı́ctor Rı́os-Rull, “Inequality in Macroeconomics,” Handbook of

Income Distribution, 2015, 2, 1229 – 1302.

Rafailov, Rafael, Archit Sharma, Eric Mitchell, Stefano Ermon, Christopher D. Manning,

and Chelsea Finn, “Direct Preference Optimization: Your Language Model is Secretly a

Reward Model,” 2024.

Raissi, M., P. Perdikaris, and G.E. Karniadakis, “Physics-informed neural networks: A deep

learning framework for solving forward and inverse problems involving nonlinear partial

differential equations,” Journal of Computational Physics, 2019, 378, 686–707.

Robbins, Herbert and Sutton Monro, “A Stochastic Approximation Method,” The Annals of

Mathematical Statistics, 1951, 22 (3), 400–407.

Rotemberg, Julio J, “Sticky prices in the United States,” Journal of political economy, 1982, 90 (6),

1187–1211.

Sargent, Thomas J., “Equilibrium with Signal Extraction from Endogenous Variables,” Journal

of Economic Dynamics and Control, 1991, 15 (2), 245–273.

, The Conquest of American Inflation, Princeton University Press, 1999.

, “HAOK and HANK Models,” Unpublished manuscript, 2023. Available at http://www.

tomsargent.com/research/HAOK\_HANK.pdf.

and John Stachurski, “Quantitative Economics with JAX,” 2025. Available at https://jax.

quantecon.org/intro.html.

Schaab, Andreas, “Micro and Macro Uncertainty,” Working Paper, UC Berkeley 2020.

Silver,

David,

Introduction to Reinforcement Learning,

DeepMind x UCL,

Video

recordings

available

at

https://www.youtube.com/playlist?list=

PLqYmG7hTraZDM-OYHWgPebj2MfCFzFObQ, 2015.

, Aja Huang, Chris J. Maddison, Arthur Guez, Laurent Sifre, George van den Driessche,

Julian Schrittwieser, Ioannis Antonoglou, Veda Panneershelvam, Marc Lanctot, Sander

Dieleman, Dominik Grewe, John Nham, Nal Kalchbrenner, Ilya Sutskever, Timothy Lillicrap, Madeleine Leach, Koray Kavukcuoglu, Thore Graepel, and Demis Hassabis, “Mastering the game of Go with deep neural networks and tree search,” Nature, Jan 2016, 529

(7587), 484–489.

, Julian Schrittwieser, Karen Simonyan, Ioannis Antonoglou, Aja Huang, Arthur Guez,

Thomas Hubert, Lucas Baker, Matthew Lai, Adrian Bolton, Yutian Chen, Timothy Lillicrap, Fan Hui, Laurent Sifre, George van den Driessche, Thore Graepel, and Demis Hassabis, “Mastering the Game of Go without Human Knowledge,” Nature, Oct 2017, 550 (7676),

354–359.

Storesletten, Kjetil, Chris Telmer, and Amir Yaron, “Asset Pricing with Idiosyncratic Risk

and Overlapping Generations,” Review of Economic Dynamics, October 2007, 10 (4), 519–548.

Sutton, Richard S. and Andrew G. Barto, Reinforcement Learning: An Introduction, MIT Press,

2018.

Wu, Zida, Mathieu Lauriere, Matthieu Geist, Olivier Pietquin, and Ankur Mehta,

“Population-aware Online Mirror Descent for Mean-Field Games with Common Noise by

Deep Reinforcement Learning,” 2025.

Xu, Ruitu, Yifei Min, Tianhao Wang, Michael I. Jordan, Zhaoran Wang, and Zhuoran

Yang, “Finding Regularized Competitive Equilibria of Heterogeneous Agent Macroeconomic Models via Reinforcement Learning,” in Francisco Ruiz, Jennifer Dy, and Jan-Willem

van de Meent, eds., Proceedings of The 26th International Conference on Artificial Intelligence and

Statistics, Vol. 206 of Proceedings of Machine Learning Research PMLR 25–27 Apr 2023, pp. 375–

407.

Yang, Yaodong, Rui Luo, Minne Li, Ming Zhou, Weinan Zhang, and Jun Wang, “Mean Field

Multi-Agent Reinforcement Learning,” in Jennifer Dy and Andreas Krause, eds., Proceedings of the 35th International Conference on Machine Learning, Vol. 80 of Proceedings of Machine

Learning Research PMLR 10–15 Jul 2018, pp. 5571–5580.

Young, Eric R., “Solving the incomplete markets model with aggregate uncertainty using the

Krusell–Smith algorithm and non-stochastic simulations,” Journal of Economic Dynamics and

Control, 2010, 34 (1), 36–41.

Zhao,

Shiyu,

Mathematical

Foundations

of

Reinforcement

Learning,

Springer

Nature,

available

at

https://github.com/MathFoundationRL/

Book-Mathematical-Foundation-of-Reinforcement-Learning, 2025.

Online Appendix

A

Model Calibration and Hyperparameters

This appendix presents additional model, calibration and implementation details for the applications we solve in Section 4 using our SRL approach. We start with the Huggett (1993)

application in Appendix A.1, move to the Krusell and Smith (1998) application in Appendix

A.2, and conclude with the HANK application in Appendix A.3.

A.1

Appendix for Huggett Application

We summarize the calibration we use in Section 4.1 in Table 2. Table 3 summarizes the hyperparameters used to solve the Huggett model with our SRL algorithm. We briefly discuss the

main choices and their rationale.

Partial equilibrium specification.

In the general equilibrium Huggett model, the interest

rate is a complicated function of the aggregate state and the cross-sectional distribution. It is

not Markov. For the partial equilibrium (PE) exercise discussed in Section 4.1, we instead take

as given an exogenous Markov law of motion for the interest rate rt and let households solve

their individual problem taking as given this process.

We model the interest rate as a mean-reverting process with a square-root volatility term.

In continuous time, this is analogous to a Cox-Ingersoll-Ross (CIR) or “Feller square-root”

process, which ensures positivity of the interest rate. In discrete time, our PE price process is

specified as

rt+1 = (1 − ρr )r + ρr rt + νr

q

max\{rt , 0\} · ε r,t

where

ε r,t ∼ N (0, 1)

where r is the long-run mean level of the interest rate, ρr its autocorrelation, and νr the innovation volatility. The parameter values we use are reported in Table 2 and are chosen so that

the unconditional distribution of interest rates as well as the implied aggregate bond holdings

in PE are broadly consistent with general equilibrium in the Huggett calibration.

The idiosyncratic income process y in PE is the same as in the GE model, a three-point

discretization of a log AR(1) with persistence ρy and volatility νy , as reported in Table 2. Thus,

the only difference between PE and GE is that in PE the household takes rt as an exogenous

Markov process, while in GE it is determined endogenously from bond market clearing.

For the numerical implementation, we discretize the PE interest rate process on a grid described in Table 3. This grid is constructed using the CIR discretization method in Farmer and

Toda (2017), which is designed for square-root processes and preserves positivity. Together

with the income grid for y, this yields a fully specified PE environment in which we can solve

Parameter

β

σ

ρy

νy

ρz

νz

B

b

r

ρr

νr

Description

Discount factor

Coefficient of relative risk aversion

Autocorrelation of labor income

Variance parameter of labor income

Persistence of AR(1) for zt (log TFP)

Volatility of AR(1) for zt (log TFP)

Total bond supply

Borrowing constraint

Mean interest rate (PE)

Autocorrelation of interest rate (PE)

Volatility of interest rate (PE)

Value

0.96

0.6

0.2

0.9

0.02

-1

0.038

0.8

0.02

Table 2: Huggett model calibration

the household problem using both our SPG algorithm and a conventional VFI method, as described in the main text.

Discretization. The individual state (b, y) consists of bond holdings b and idiosyncratic income y. We discretize bonds on a one-dimensional grid with nb = 200 points and an upper

bound bmax = 50. The income process y takes ny = 3 possible values. On the aggregate side,

the two key state variables are the interest rate rt and aggregate income zt . We approximate

rt on a grid with nr = 20 points covering the interval [r L , r H ] = [0.01, 0.06]. This range is

chosen to comfortably contain all equilibrium interest rate realizations observed in our simulations while avoiding an unnecessarily large grid. Aggregate productivity zt is discretized on

a grid with nz = 30 points using a standard Tauchen procedure. These choices strike a balance between accuracy and computational cost. They are fine enough to capture the relevant

curvature in individual policies and the dependence of prices on the aggregate state.

Simulation horizon and truncation. We approximate lifetime utility by truncating the infinite sum in (15) at a finite horizon Ttrunc . We choose Ttrunc to ensure that the tail of the

discounted utility is negligible relative to a user-specified tolerance level etrunc ,

n

o

Ttrunc = min T : β T < etrunc .

In the baseline Huggett experiment, we use etrunc = 10−3 and obtain Ttrunc = 170, i.e. the

contribution of periods beyond Ttrunc is bounded by 10−3 in present-value terms. To avoid

numerical issues when wealth is very low, we also impose a minimal consumption floor cmin =

10−3 . This has no discernible effect on the economic results but prevents the utility function

from being evaluated at (or extremely close to) zero.

Training schedule and learning rate. We train the SPG algorithm with an exponentially decaying learning rate. Let lrini denote the initial learning rate and lrdecay ∈ (0, 1) the decay

Parameter

nb

bmax

ny

nr

rL

rH

nz

cmin

Ttrunc

etrunc

Nepoch

Nwarm-up

lrini

lrdecay

lrsche

Nsample

econverge

Description

Number of b grid points

Upper bound of b grid

Number of y grid points

Number of r grid points

Lower bound of r grid

Upper bound of r grid

Number of z grid points

Minimum consumption

Truncation horizon for simulations

Truncation threshold

Maximum number of parameter updates

Number of warm-up epochs

Initial learning rate

Learning rate decay rate

Learning-rate scheduler

Batch size (trajectories per update)

Convergence threshold

Value

200

20

0.01

0.06

30

10−3

170

10−3

1000

50

10−3

0.5

exponential

512

3 × 10−4

Table 3: Hyperparameters for solving the Huggett model

factor. The learning rate at iteration t is given by

t

lrt = lrini · lrdecay

,

where

t0 =

max\{t − Nwarm-up , 0\}

,

Nepoch − Nwarm-up

so that lrt is held constant during an initial “warm-up” phase of length Nwarm-up and then

decays smoothly to a lower value by the final epoch Nepoch . In the Huggett application, we

set Nepoch = 1000, Nwarm-up = 50, lrini = 10−3 , and lrdecay = 0.5, and we use an exponential

scheduler (denoted by lrsche in Table 3). We declare convergence when the change in the policy

parameters across epochs falls below the threshold econverge = 3 × 10−4 .

Sampling, batching, and memory constraints.

Due to GPU memory constraints, we do not

use all simulated data to update the policy in each iteration. Instead, we sample data in minibatches. In each update, the effective data size is Nsample × Nupdate , where Nsample denotes

the number of simulated trajectories per batch (we set Nsample = 512 in the baseline Huggett

experiment) and Nupdate the number of time steps used from each trajectory for the gradient

update. This mini-batching keeps memory requirements manageable while preserving enough

variation in the data to obtain stable gradient estimates.

Initialization and warm-up. We initialize the policy as described in Footnote 20 to guarantee

that the initial aggregate savings schedule is at least weakly responsive to the interest rate. The

training process is then split into two phases. During the warm-up phase of length Nwarm-up , we

fix the cross-sectional distribution of agents at some simple initial guess g0 and do not update

it. In this phase, the sole objective is to move the policy away from its crude initial guess

and toward a reasonable neighborhood of the eventual solution. Keeping g fixed prevents the

Parameter

β

σ

b

ρy

νy

a

δ

ρz

νz

Description

Discount factor

Utility parameter

Borrowing constraint

Autocorrelation of idiosyncratic shock

Volatility of idiosyncratic shock

Capital share

Capital depreciation rate

Persistence of AR(1) for zt (log TFP)

Volatility of AR(1) for zt (log TFP)

Value

0.95

0.6

0.2

0.36

0.08

0.9

0.03

Table 4: Krusell–Smith model calibration

badly informed initial policy from “polluting” the distribution.

After warm-up, we switch to an adaptive phase in which the distribution is updated endogenously, and which may last up to Nepoch − Nwarm-up epochs ( though convergence typically

occurs earlier). We use the simulated distribution implied by the most recent policy as the

initial distribution for each trajectory. In other words, after warm-up, each new batch of trajectories starts from a cross-section that is itself an equilibrium object. This iterative updating of

the initial distribution ensures that the policy is trained on data drawn from its own induced

stationary distribution, which is important for the accuracy of the final solution.

A.2

Appendix for Krusell-Smith Application

Calibration. Table 4 summarizes the calibration for the Krusell-Smith model used in Section

4.2. We use a discount factor of β = 0.95. The utility function is CRRA with a coefficient of

relative risk aversion σ = 3, and the borrowing constraint is set at b = 0.

The idiosyncratic income process is modeled as a log AR(1) with autocorrelation ρy = 0.6

and innovation volatility νy = 0.2, the same specification as in the Huggett model. On the

production side, we follow the standard Krusell-Smith calibration: the capital share is α =

0.36, the depreciation rate is δ = 0.08, and aggregate productivity zt follows a log AR(1) with

persistence ρz = 0.9 and volatility νz = 0.03.

Discretization and grids.

Table 5 reports the hyperparameters used for the SPG solution of

the Krusell-Smith model. The individual capital state b is discretized on a grid with nb = 200

points and an upper bound bmax = 100, which is higher than in the Huggett experiment. The

larger upper bound reflects the fact that, in Krusell-Smith, agents accumulate capital rather

than unproductive bonds, and the equilibrium wealth distribution is more dispersed. The

idiosyncratic income state y again takes ny = 3 values.

The aggregate price vector consists of the interest rate rt and the real wage wt . We approximate the price space on two separate grids: the interest rate is discretized with nr = 30 points

on [r L , r H ] = [0.02, 0.07], and the wage with nw = 50 points on [w L , w H ] = [0.9, 1.5]. These

ranges comfortably contain the realizations observed in our simulations and allow the policy

to respond flexibly to movements in both prices without requiring a prohibitive number of

Parameter

nb

bmax

ny

nr

rL

rH

nw

wL

wH

cmin

cinit

Ttrunc

etrunc

Nepoch

Nwarm-up

lrini

lrdecay

lrsche

Nsample

econverge

Description

Number of b grid points

Upper bound of b grid

Number of y grid points

Number of r grid points

Lower bound of r grid

Upper bound of r grid

Number of p2 grid points

Lower bound of w grid

Upper bound of w grid

Minimum consumption

Initial guess for consumption share

Truncation horizon for simulations

Truncation threshold

Maximum number of parameter updates

Number of warm-up epochs

Initial learning rate

Learning-rate decay rate

Learning-rate scheduler

Batch size (trajectories per update)

Convergence threshold

Value

200

100

30

0.02

0.07

0.9

1.5

10−3

0.5

90

10−2

1000

50

5 × 10−4

0.5

exponential

512

2 × 10−4

Table 5: Hyperparameters for solving the Krusell–Smith model

grid points.

Simulation horizon and truncation.

As in the Huggett case, lifetime utility is computed by

truncating the infinite sum at a finite horizon Ttrunc . For Krusell-Smith we set Ttrunc = 90 and

use a truncation tolerance etrunc = 10−2 , i.e.

β Ttrunc < etrunc ,

so that the tail of the discounted utility stream is negligible at the scale of our numerical accuracy. We also impose a minimal consumption level cmin = 10−3 to avoid evaluating utility at

zero or extremely small consumption levels.

Training schedule and convergence.

We use the same general training structure as in the

Huggett exercise but with slightly different numerical values. The maximum number of epochs

is Nepoch = 1000, with Nwarm-up = 50 warm-up epochs during which the learning rate is kept

constant and the initial distribution is fixed. The learning rate starts at lrini = 5 × 10−4 and

decays exponentially at rate lrdecay = 0.5 according to the scheduler denoted by lrsche in

Table 5. As in the Huggett case, the decay is only activated after the warm-up phase. We

declare convergence when the change in parameters between successive epochs falls below

econverge = 2 × 10−4 ; in practice, the algorithm typically converges well before hitting the hard

cap of Nepoch .

Mini-batching is again used to handle memory constraints and stabilize gradient estimates.

Parameter

β

σ

η

φ

θ

e

R̄

ρz

νz

ρe

νe

Description

Discount factor

Coefficient of relative risk aversion

Inverse of Frisch elasticity

Coefficient of Taylor rule

Price adjustment cost

Elasticity of substitution

Target for gross interest rate

Autocorrelation of aggregate TFP shock

Volatility of aggregate TFP shock

Autocorrelation of monetary policy shock

Volatility of monetary policy shock

Value

0.975

1.5

100

10

1.025

0.9

0.07

0.9

0.002

Table 6: Calibration for the HANK model

Each update uses Nsample = 512 simulated trajectories, and a fixed number of time steps

per trajectory, to form the stochastic gradient. This yields a total data size per update of

Nsample × Nupdate , which we choose to fully utilize the available GPU memory without inducing excessive variance in the gradient.

Initialization and warm-up.

The warm-up logic mirrors that used in the Huggett application

but is adapted to the two-price environment. During the first Nwarm-up epochs, all trajectories

are initialized from a cross-sectional distribution that is held fixed, while the policy is being

updated. After warm-up, we update the cross-sectional distribution based on the most recent policy and allow the learning rate to adjust. The initial conditions for each new set of

trajectories are drawn from the simulated distribution generated by the most recent policy.

A.3

Appendix for HANK Application

Calibration. Table 6 reports the calibration of the HANK model used in Section 4.3. One

model period corresponds to a year, and the discount factor is set such that the annual real

interest rate is in a plausible range; in the baseline we use a discount factor of 0.975. Preferences

over consumption and labor are CRRA and separable, with coefficient of relative risk aversion

equal to 1 and inverse Frisch elasticity of labor supply η = 1.

We follow a standard New Keynesian calibration. Price-setting firms face Rotemberg adjustment costs with parameter θ = 100 and an elasticity of substitution across intermediate

goods of ε = 10. Monetary policy follows the Taylor rule specified in the main text, with

coefficient φ = 1.5 on inflation and a gross steady-state real interest rate target R̄ = 1.025.

Aggregate risk is two-dimensional. TFP zt follows an AR(1) process with persistence ρz =

0.9 and innovation volatility νz = 0.07. The monetary policy shock et is also AR(1) with the

same persistence ρe = 0.9 and volatility νe = 0.002. These values are summarized in Table 6

and are chosen so that both real and nominal variables display non-trivial but stable dynamics

in the simulations.

Parameter

nb

bmax

y

nr

rL

rH

nw

wL

wH

nz

ne

cmin

cinit

ninit

Πinit

Ttrunc

etrunc

Nepoch

Nwarm-up

lrini

lrdecay

Nsample

econverge

Description

Number of b grid points

Upper bound of b grid

Number of y grid points

Number of r grid points

Lower bound of r grid

Upper bound of r grid

Number of p2 grid points

Lower bound of w grid

Upper bound of w grid

Number of z grid points

Number of e grid points

Minimum consumption

Initial guess of consumption share

Initial guess of labor supply

Initial guess of inflation

Truncation horizon for simulations

Truncation threshold

Maximum number of parameter updates

Number of warm-up epochs

Initial learning rate

Learning-rate decay rate

Baseline sampling size

Convergence threshold

Value

200

100

30

0.01

0.04

0.7

1.0

50

10−3

0.5

1.5

182

10−2

1000

50

5 × 10−3

0.5

512

2 × 10−3

Table 7: Hyperparameters for solving the HANK model

Discretization and grids. Table 7 lists the hyperparameters and grid choices for our solution

of the HANK model. The individual asset state b is discretized on a grid with nb = 200 points

and an upper bound bmax = 100. The idiosyncratic income state y again takes ny = 3 values,

using the same discretization as in the Huggett and Krusell-Smith applications for ease of

comparison.

The aggregate state combines a two-dimensional price vector (rt , wt ) with the two exogenous shocks zt and et . We approximate the real interest rate on a grid with nr = 30 points

over the interval [r L , r H ] = [0.01, 0.04]. The real wage is discretized with nw = 30 points on

[w L , w H ] = [0.7, 1.0]. The ranges are chosen to cover comfortably the realizations observed in

equilibrium simulations, while keeping the price grids small enough for efficient training.

For the aggregate shocks, we use nz = 50 grid points for log TFP zt and ne = 50 points

for the monetary shock et . Both grids are obtained by discretizing the respective AR(1) processes with a standard Tauchen method. The relatively fine grids for (zt , et ) help the algorithm

capture the interaction between real and nominal disturbances in the HANK model.

Simulation horizon, truncation, and initial guesses. The HANK model combines persistence in both TFP and monetary shocks with sluggish price adjustment. To accommodate this

accurately, we use a longer truncation horizon Ttrunc = 182 periods and a truncation tolerance

etrunc = 10−2 , which implies

β Ttrunc < etrunc ,

so that the contribution of periods beyond Ttrunc is negligible at the scale of our numerical

accuracy. As in the other applications, we impose a minimal consumption level cmin = 10−3 to

avoid evaluating the utility function at zero consumption.

The HANK environment includes both consumption-saving and labor-supply decisions,

as well as firm price-setting. We initialize these policy functions using simple guess rules;

for cinit , we set an initial constant consumption share of total cash-in-hand, a constant ninit

sets an initial level of hours worked, and Πinit = 0 initializes inflation. These initial values

are deliberately crude and their sole purpose is to place the policy in a reasonable region of

the parameter space before learning from simulated data. In practice, the final solution is

insensitive to these initial guesses once training has converged.

Training and convergence. The remaining training hyperparameters follow the logic of the

Huggett and Krusell-Smith applications. We use a baseline batch size of Nsample = 512 simulated trajectories per update, chosen to saturate GPU memory without generating excessive

variance in the policy gradient. The convergence threshold is set at econverge = 2 × 10−3 for

the HANK model, which reflects the greater complexity of the joint household-firm problem

and the fact that small changes in the policy parameters can translate into larger differences in

aggregate dynamics.

Other aspects of the training schedule — notably the total number of epochs, the length of

the warm-up phase, and the learning-rate schedule — are chosen in line with the Krusell-Smith

specification discussed in Appendix A.2 and are not repeated here. In practice, the HANK

model converges somewhat more slowly than Huggett and Krusell-Smith but still within a

few minutes on a single GPU, as reported in Table 1 in the main text.


\end{document}