Online Appendix

A

Model Calibration and Hyperparameters

This appendix presents additional model, calibration and implementation details for the applications we solve in Section 4 using our SRL approach. We start with the Huggett (1993)

application in Appendix A.1, move to the Krusell and Smith (1998) application in Appendix

A.2, and conclude with the HANK application in Appendix A.3.

A.1

Appendix for Huggett Application

We summarize the calibration we use in Section 4.1 in Table 2. Table 3 summarizes the hyperparameters used to solve the Huggett model with our SRL algorithm. We briefly discuss the

main choices and their rationale.

Partial equilibrium specification.

In the general equilibrium Huggett model, the interest

rate is a complicated function of the aggregate state and the cross-sectional distribution. It is

not Markov. For the partial equilibrium (PE) exercise discussed in Section 4.1, we instead take

as given an exogenous Markov law of motion for the interest rate rt and let households solve

their individual problem taking as given this process.

We model the interest rate as a mean-reverting process with a square-root volatility term.

In continuous time, this is analogous to a Cox-Ingersoll-Ross (CIR) or “Feller square-root”

process, which ensures positivity of the interest rate. In discrete time, our PE price process is

specified as

rt+1 = (1 − ρr )r + ρr rt + νr

q

max\{rt , 0\} · ε r,t

where

ε r,t ∼ N (0, 1)

where r is the long-run mean level of the interest rate, ρr its autocorrelation, and νr the innovation volatility. The parameter values we use are reported in Table 2 and are chosen so that

the unconditional distribution of interest rates as well as the implied aggregate bond holdings

in PE are broadly consistent with general equilibrium in the Huggett calibration.

The idiosyncratic income process y in PE is the same as in the GE model, a three-point

discretization of a log AR(1) with persistence ρy and volatility νy , as reported in Table 2. Thus,

the only difference between PE and GE is that in PE the household takes rt as an exogenous

Markov process, while in GE it is determined endogenously from bond market clearing.

For the numerical implementation, we discretize the PE interest rate process on a grid described in Table 3. This grid is constructed using the CIR discretization method in Farmer and

Toda (2017), which is designed for square-root processes and preserves positivity. Together

with the income grid for y, this yields a fully specified PE environment in which we can solve

Parameter

β

σ

ρy

νy

ρz

νz

B

b

r

ρr

νr

Description

Discount factor

Coefficient of relative risk aversion

Autocorrelation of labor income

Variance parameter of labor income

Persistence of AR(1) for zt (log TFP)

Volatility of AR(1) for zt (log TFP)

Total bond supply

Borrowing constraint

Mean interest rate (PE)

Autocorrelation of interest rate (PE)

Volatility of interest rate (PE)

Value

0.96

0.6

0.2

0.9

0.02

-1

0.038

0.8

0.02

Table 2: Huggett model calibration

the household problem using both our SPG algorithm and a conventional VFI method, as described in the main text.

Discretization. The individual state (b, y) consists of bond holdings b and idiosyncratic income y. We discretize bonds on a one-dimensional grid with nb = 200 points and an upper

bound bmax = 50. The income process y takes ny = 3 possible values. On the aggregate side,

the two key state variables are the interest rate rt and aggregate income zt . We approximate

rt on a grid with nr = 20 points covering the interval [r L , r H ] = [0.01, 0.06]. This range is

chosen to comfortably contain all equilibrium interest rate realizations observed in our simulations while avoiding an unnecessarily large grid. Aggregate productivity zt is discretized on

a grid with nz = 30 points using a standard Tauchen procedure. These choices strike a balance between accuracy and computational cost. They are fine enough to capture the relevant

curvature in individual policies and the dependence of prices on the aggregate state.

Simulation horizon and truncation. We approximate lifetime utility by truncating the infinite sum in (15) at a finite horizon Ttrunc . We choose Ttrunc to ensure that the tail of the

discounted utility is negligible relative to a user-specified tolerance level etrunc ,

n

o

Ttrunc = min T : β T < etrunc .

In the baseline Huggett experiment, we use etrunc = 10−3 and obtain Ttrunc = 170, i.e. the

contribution of periods beyond Ttrunc is bounded by 10−3 in present-value terms. To avoid

numerical issues when wealth is very low, we also impose a minimal consumption floor cmin =

10−3 . This has no discernible effect on the economic results but prevents the utility function

from being evaluated at (or extremely close to) zero.

Training schedule and learning rate. We train the SPG algorithm with an exponentially decaying learning rate. Let lrini denote the initial learning rate and lrdecay ∈ (0, 1) the decay

Parameter

nb

bmax

ny

nr

rL

rH

nz

cmin

Ttrunc

etrunc

Nepoch

Nwarm-up

lrini

lrdecay

lrsche

Nsample

econverge

Description

Number of b grid points

Upper bound of b grid

Number of y grid points

Number of r grid points

Lower bound of r grid

Upper bound of r grid

Number of z grid points

Minimum consumption

Truncation horizon for simulations

Truncation threshold

Maximum number of parameter updates

Number of warm-up epochs

Initial learning rate

Learning rate decay rate

Learning-rate scheduler

Batch size (trajectories per update)

Convergence threshold

Value

200

20

0.01

0.06

30

10−3

170

10−3

1000

50

10−3

0.5

exponential

512

3 × 10−4

Table 3: Hyperparameters for solving the Huggett model

factor. The learning rate at iteration t is given by

t

lrt = lrini · lrdecay

,

where

t0 =

max\{t − Nwarm-up , 0\}

,

Nepoch − Nwarm-up

so that lrt is held constant during an initial “warm-up” phase of length Nwarm-up and then

decays smoothly to a lower value by the final epoch Nepoch . In the Huggett application, we

set Nepoch = 1000, Nwarm-up = 50, lrini = 10−3 , and lrdecay = 0.5, and we use an exponential

scheduler (denoted by lrsche in Table 3). We declare convergence when the change in the policy

parameters across epochs falls below the threshold econverge = 3 × 10−4 .

Sampling, batching, and memory constraints.

Due to GPU memory constraints, we do not

use all simulated data to update the policy in each iteration. Instead, we sample data in minibatches. In each update, the effective data size is Nsample × Nupdate , where Nsample denotes

the number of simulated trajectories per batch (we set Nsample = 512 in the baseline Huggett

experiment) and Nupdate the number of time steps used from each trajectory for the gradient

update. This mini-batching keeps memory requirements manageable while preserving enough

variation in the data to obtain stable gradient estimates.

Initialization and warm-up. We initialize the policy as described in Footnote 20 to guarantee

that the initial aggregate savings schedule is at least weakly responsive to the interest rate. The

training process is then split into two phases. During the warm-up phase of length Nwarm-up , we

fix the cross-sectional distribution of agents at some simple initial guess g0 and do not update

it. In this phase, the sole objective is to move the policy away from its crude initial guess

and toward a reasonable neighborhood of the eventual solution. Keeping g fixed prevents the

Parameter

β

σ

b

ρy

νy

a

δ

ρz

νz

Description

Discount factor

Utility parameter

Borrowing constraint

Autocorrelation of idiosyncratic shock

Volatility of idiosyncratic shock

Capital share

Capital depreciation rate

Persistence of AR(1) for zt (log TFP)

Volatility of AR(1) for zt (log TFP)

Value

0.95

0.6

0.2

0.36

0.08

0.9

0.03

Table 4: Krusell–Smith model calibration

badly informed initial policy from “polluting” the distribution.

After warm-up, we switch to an adaptive phase in which the distribution is updated endogenously, and which may last up to Nepoch − Nwarm-up epochs ( though convergence typically

occurs earlier). We use the simulated distribution implied by the most recent policy as the

initial distribution for each trajectory. In other words, after warm-up, each new batch of trajectories starts from a cross-section that is itself an equilibrium object. This iterative updating of

the initial distribution ensures that the policy is trained on data drawn from its own induced

stationary distribution, which is important for the accuracy of the final solution.

A.2

Appendix for Krusell-Smith Application

Calibration. Table 4 summarizes the calibration for the Krusell-Smith model used in Section

4.2. We use a discount factor of β = 0.95. The utility function is CRRA with a coefficient of

relative risk aversion σ = 3, and the borrowing constraint is set at b = 0.

The idiosyncratic income process is modeled as a log AR(1) with autocorrelation ρy = 0.6

and innovation volatility νy = 0.2, the same specification as in the Huggett model. On the

production side, we follow the standard Krusell-Smith calibration: the capital share is α =

0.36, the depreciation rate is δ = 0.08, and aggregate productivity zt follows a log AR(1) with

persistence ρz = 0.9 and volatility νz = 0.03.

Discretization and grids.

Table 5 reports the hyperparameters used for the SPG solution of

the Krusell-Smith model. The individual capital state b is discretized on a grid with nb = 200

points and an upper bound bmax = 100, which is higher than in the Huggett experiment. The

larger upper bound reflects the fact that, in Krusell-Smith, agents accumulate capital rather

than unproductive bonds, and the equilibrium wealth distribution is more dispersed. The

idiosyncratic income state y again takes ny = 3 values.

The aggregate price vector consists of the interest rate rt and the real wage wt . We approximate the price space on two separate grids: the interest rate is discretized with nr = 30 points

on [r L , r H ] = [0.02, 0.07], and the wage with nw = 50 points on [w L , w H ] = [0.9, 1.5]. These

ranges comfortably contain the realizations observed in our simulations and allow the policy

to respond flexibly to movements in both prices without requiring a prohibitive number of

Parameter

nb

bmax

ny

nr

rL

rH

nw

wL

wH

cmin

cinit

Ttrunc

etrunc

Nepoch

Nwarm-up

lrini

lrdecay

lrsche

Nsample

econverge

Description

Number of b grid points

Upper bound of b grid

Number of y grid points

Number of r grid points

Lower bound of r grid

Upper bound of r grid

Number of p2 grid points

Lower bound of w grid

Upper bound of w grid

Minimum consumption

Initial guess for consumption share

Truncation horizon for simulations

Truncation threshold

Maximum number of parameter updates

Number of warm-up epochs

Initial learning rate

Learning-rate decay rate

Learning-rate scheduler

Batch size (trajectories per update)

Convergence threshold

Value

200

100

30

0.02

0.07

0.9

1.5

10−3

0.5

90

10−2

1000

50

5 × 10−4

0.5

exponential

512

2 × 10−4

Table 5: Hyperparameters for solving the Krusell–Smith model

grid points.

Simulation horizon and truncation.

As in the Huggett case, lifetime utility is computed by

truncating the infinite sum at a finite horizon Ttrunc . For Krusell-Smith we set Ttrunc = 90 and

use a truncation tolerance etrunc = 10−2 , i.e.

β Ttrunc < etrunc ,

so that the tail of the discounted utility stream is negligible at the scale of our numerical accuracy. We also impose a minimal consumption level cmin = 10−3 to avoid evaluating utility at

zero or extremely small consumption levels.

Training schedule and convergence.

We use the same general training structure as in the

Huggett exercise but with slightly different numerical values. The maximum number of epochs

is Nepoch = 1000, with Nwarm-up = 50 warm-up epochs during which the learning rate is kept

constant and the initial distribution is fixed. The learning rate starts at lrini = 5 × 10−4 and

decays exponentially at rate lrdecay = 0.5 according to the scheduler denoted by lrsche in

Table 5. As in the Huggett case, the decay is only activated after the warm-up phase. We

declare convergence when the change in parameters between successive epochs falls below

econverge = 2 × 10−4 ; in practice, the algorithm typically converges well before hitting the hard

cap of Nepoch .

Mini-batching is again used to handle memory constraints and stabilize gradient estimates.

Parameter

β

σ

η

φ

θ

e

R̄

ρz

νz

ρe

νe

Description

Discount factor

Coefficient of relative risk aversion

Inverse of Frisch elasticity

Coefficient of Taylor rule

Price adjustment cost

Elasticity of substitution

Target for gross interest rate

Autocorrelation of aggregate TFP shock

Volatility of aggregate TFP shock

Autocorrelation of monetary policy shock

Volatility of monetary policy shock

Value

0.975

1.5

100

10

1.025

0.9

0.07

0.9

0.002

Table 6: Calibration for the HANK model

Each update uses Nsample = 512 simulated trajectories, and a fixed number of time steps

per trajectory, to form the stochastic gradient. This yields a total data size per update of

Nsample × Nupdate , which we choose to fully utilize the available GPU memory without inducing excessive variance in the gradient.

Initialization and warm-up.

The warm-up logic mirrors that used in the Huggett application

but is adapted to the two-price environment. During the first Nwarm-up epochs, all trajectories

are initialized from a cross-sectional distribution that is held fixed, while the policy is being

updated. After warm-up, we update the cross-sectional distribution based on the most recent policy and allow the learning rate to adjust. The initial conditions for each new set of

trajectories are drawn from the simulated distribution generated by the most recent policy.

A.3

Appendix for HANK Application

Calibration. Table 6 reports the calibration of the HANK model used in Section 4.3. One

model period corresponds to a year, and the discount factor is set such that the annual real

interest rate is in a plausible range; in the baseline we use a discount factor of 0.975. Preferences

over consumption and labor are CRRA and separable, with coefficient of relative risk aversion

equal to 1 and inverse Frisch elasticity of labor supply η = 1.

We follow a standard New Keynesian calibration. Price-setting firms face Rotemberg adjustment costs with parameter θ = 100 and an elasticity of substitution across intermediate

goods of ε = 10. Monetary policy follows the Taylor rule specified in the main text, with

coefficient φ = 1.5 on inflation and a gross steady-state real interest rate target R̄ = 1.025.

Aggregate risk is two-dimensional. TFP zt follows an AR(1) process with persistence ρz =

0.9 and innovation volatility νz = 0.07. The monetary policy shock et is also AR(1) with the

same persistence ρe = 0.9 and volatility νe = 0.002. These values are summarized in Table 6

and are chosen so that both real and nominal variables display non-trivial but stable dynamics

in the simulations.

Parameter

nb

bmax

y

nr

rL

rH

nw

wL

wH

nz

ne

cmin

cinit

ninit

Πinit

Ttrunc

etrunc

Nepoch

Nwarm-up

lrini

lrdecay

Nsample

econverge

Description

Number of b grid points

Upper bound of b grid

Number of y grid points

Number of r grid points

Lower bound of r grid

Upper bound of r grid

Number of p2 grid points

Lower bound of w grid

Upper bound of w grid

Number of z grid points

Number of e grid points

Minimum consumption

Initial guess of consumption share

Initial guess of labor supply

Initial guess of inflation

Truncation horizon for simulations

Truncation threshold

Maximum number of parameter updates

Number of warm-up epochs

Initial learning rate

Learning-rate decay rate

Baseline sampling size

Convergence threshold

Value

200

100

30

0.01

0.04

0.7

1.0

50

10−3

0.5

1.5

182

10−2

1000

50

5 × 10−3

0.5

512

2 × 10−3

Table 7: Hyperparameters for solving the HANK model

Discretization and grids. Table 7 lists the hyperparameters and grid choices for our solution

of the HANK model. The individual asset state b is discretized on a grid with nb = 200 points

and an upper bound bmax = 100. The idiosyncratic income state y again takes ny = 3 values,

using the same discretization as in the Huggett and Krusell-Smith applications for ease of

comparison.

The aggregate state combines a two-dimensional price vector (rt , wt ) with the two exogenous shocks zt and et . We approximate the real interest rate on a grid with nr = 30 points

over the interval [r L , r H ] = [0.01, 0.04]. The real wage is discretized with nw = 30 points on

[w L , w H ] = [0.7, 1.0]. The ranges are chosen to cover comfortably the realizations observed in

equilibrium simulations, while keeping the price grids small enough for efficient training.

For the aggregate shocks, we use nz = 50 grid points for log TFP zt and ne = 50 points

for the monetary shock et . Both grids are obtained by discretizing the respective AR(1) processes with a standard Tauchen method. The relatively fine grids for (zt , et ) help the algorithm

capture the interaction between real and nominal disturbances in the HANK model.

Simulation horizon, truncation, and initial guesses. The HANK model combines persistence in both TFP and monetary shocks with sluggish price adjustment. To accommodate this

accurately, we use a longer truncation horizon Ttrunc = 182 periods and a truncation tolerance

etrunc = 10−2 , which implies

β Ttrunc < etrunc ,

so that the contribution of periods beyond Ttrunc is negligible at the scale of our numerical

accuracy. As in the other applications, we impose a minimal consumption level cmin = 10−3 to

avoid evaluating the utility function at zero consumption.

The HANK environment includes both consumption-saving and labor-supply decisions,

as well as firm price-setting. We initialize these policy functions using simple guess rules;

for cinit , we set an initial constant consumption share of total cash-in-hand, a constant ninit

sets an initial level of hours worked, and Πinit = 0 initializes inflation. These initial values

are deliberately crude and their sole purpose is to place the policy in a reasonable region of

the parameter space before learning from simulated data. In practice, the final solution is

insensitive to these initial guesses once training has converged.

Training and convergence. The remaining training hyperparameters follow the logic of the

Huggett and Krusell-Smith applications. We use a baseline batch size of Nsample = 512 simulated trajectories per update, chosen to saturate GPU memory without generating excessive

variance in the policy gradient. The convergence threshold is set at econverge = 2 × 10−3 for

the HANK model, which reflects the greater complexity of the joint household-firm problem

and the fact that small changes in the policy parameters can translate into larger differences in

aggregate dynamics.

Other aspects of the training schedule — notably the total number of epochs, the length of

the warm-up phase, and the learning-rate schedule — are chosen in line with the Krusell-Smith

specification discussed in Appendix A.2 and are not repeated here. In practice, the HANK

model converges somewhat more slowly than Huggett and Krusell-Smith but still within a

few minutes on a single GPU, as reported in Table 1 in the main text.