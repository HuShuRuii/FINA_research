
\section{Conclusion}

We develop a new structural reinforcement learning (SRL) approach to formulating and globally solving heterogeneous agent models with aggregate risk. We replace the cross-sectional

distribution with low-dimensional prices as state variables and let agents compute price expectations directly from simulated paths. Our approach differs from standard RL in that we

assume that agents have structural knowledge about the dynamics of their own individual

states (e.g. their budget constraint and idiosyncratic income process). Our structural policy

gradient (SPG) algorithm sidesteps the Master equation and efficiently handles heterogeneous

agent models traditional methods struggle with, like those with nontrivial market-clearing

conditions. By imposing that policy functions depend only on current prices (or a short price

history) we keep the state space low-dimensional so that we can work with a grid-based (tabular) approach rather than deep neural networks.

We implement our method in JAX and conduct computational experiments for three benchmark models in macroeconomics – the Huggett (1993) model, the Krusell and Smith (1998)

model, and a one-asset HANK model with sticky prices. In all three cases, the SPG algorithm

converges in only a few minutes. In the Krusell-Smith model, the resulting policies are close

to those obtained from alternative global solutions of the rational expectations equilibrium.

Allowing agents to condition on a longer history of lagged prices hardly moves the solution,

indicating that much of the relevant information for forecasting future prices is already contained in current prices. In the HANK model, we show how the same approach can be used

symmetrically on the household and firm sides to globally solve the forward-looking New

Keynesian Phillips curve.

The core idea of our approach – that agents form price expectations by sampling – could, in

principle, serve as a building block for an empirically realistic theory of expectations formation

in macroeconomics (which the algorithm in this paper is not). The plausibility of RL-based approaches is supported by evidence that reinforcement learning underpins a substantial share

of human and animal learning.26 To develop such an RL-based theory of expectations formation would require several modifications to our SRL method. First, the algorithm would need

to be converted to a fully online, incremental RL algorithm, with agents updating policies

and value estimates continuously while interacting with their environment, rather than only

after observing N price trajectories of length T. Second, the assumption that agents sample

equilibrium prices in an unbiased way would likely need to be relaxed. Empirical evidence

suggests that people disproportionately weight certain personal experiences (e.g. Malmendier

and Nagel, 2011, 2015), so incorporating biased or experience-weighted sampling would be

more realistic. Third, our SRL agents form price expectations in a model-free manner; this

may be too simplistic and one could instead treat prices using a model-based RL approach.

